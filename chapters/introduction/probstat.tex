Using \gls{inp} to keep scaling data centers' performance seems a promising idea: Daiet \cite{daiet} inventors claim to achieve a 86.9\%-89.3\% traffic reduction, hence reducing servers' workload; NetChain \cite{netchain} can process queries entirely in the network data plane, eliminating the query processing at servers and cutting the end-to-end latency to as little as half of an RTT.\\
Current data center \glspl{rm} (e.g., Apache YARN \cite{yarn}, Google Omega \cite{omega}) are not completely network-unaware: for instance, some of them are capable of satisfying affinity rules. CloudMirror \cite{cloudmirror} even provides bandwidth guarantees to tenant applications. Still, current \glspl{rm} do not consider \gls{inp} resources.\\
As a consequence, tenant applications cannot request \gls{inp} services while asking for server resources.

\subsection{Modeling \texorpdfstring{\glsentryshort{inp}}{INP} resources}
This Master's thesis goal consists in investigating how to model \gls{inp} resources and how to integrate them in RMs.\par
In order to offer \gls{inp} services to a tenant application, the latter should be capable of asking for \gls{inp} resources through an \gls{api}. To do that, \gls{inp} resources must be modeled not only to support currently existing \gls{inp} solutions such as \cite{daiet} \cite{netchain} \cite{incbricks} \cite{sharp}, but also to support future ones. It might be convenient to derive a single model to describe both server and \gls{inp} resources.\par
Classic tenant application requests can often be modeled as a key-value data structure. CloudMirror \cite{cloudmirror} requires a \gls{tag} as an input, which is a directed graph where each vertex represents an application component and links' weights represent the minimum requested bandwidth. One possible model could be based on a \gls{tag}, describing network resources and/or \gls{inp} services as vertexes or links. Tenants applications could either use the same model used within the data center or a simplified one, adding another level of abstraction.\par
Finally, a network-aware placement algorithm in the Resource Manager should then be able to allocate the requested resources accordingly.