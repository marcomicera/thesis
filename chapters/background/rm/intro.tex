In a data center, resources of any kind are being virtualized in order to achieve higher flexibility, portability and availability.
Usually both compute and storage resources are virtualized by means of \glspl{vm} and containers.
Flexibility and portability are both automatically achieved thanks to this resource virtualization, resulting in some software that can be deployed dynamically, run by multiple platforms and even live migrated; availability is usually simply achieved by not co-locating \glspl{vm} in a single server or rack.

Furthermore, a Resource Manager's aim is to manage all resources in a cluster and to schedule applications (sometimes referred as \textit{jobs}) assigning the corresponding \glspl{vm}/containers to the \textit{best} subset of servers.

Today there are multiple Resource Managers that are using different approaches to solve different design issues.
This section examines these existing \glspl{rm}, trying to categorize them based on how they face different scheduling problems.