In the past, network operators used to run network functions on middleboxes: nowadays, commodity hardware is being deployed more and more since it is cheaper and provides more flexibility by running network functions as a software (VNF). This higher flexibility can allow network operators to fully utilize their commodity hardware by offering processing on demand to other parties, effectively becoming miniature cloud providers specialized for in-network processing. In-Net \cite{in-net} is an architecture that allows untrusted endpoints to deploy custom in-network processing to be run on such devices.\\
The main reason why existing cloud solutions cannot simply be adopted to support INP platforms is scalability: commodity hardware are not able to run as many VMs as needed. As a consequence, this solution lets a single VM serve multiple users in a secure manner.\\
Besides preventing malicious users to use providers' commodity hardware to perform large-scale DDoS attacks, In-Net \cite{in-net} makes use of symbolic execution (in two configurations: logic implemented in the network and on a server as usual) to determine whether custom INP tenant applications are secure or not. A network controller instantiates the tenant application on a commodity hardware and then the checking is done.