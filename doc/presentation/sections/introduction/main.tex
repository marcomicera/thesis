\begin{frame}[fragile]{Introduction}
    \inputifexists{sections/introduction/beamernotes/introduction.tex}

    % \vspace{3mm}

    % HPC, thousands of requests
    \begin{aquote}{NetChain\footnotemark{} authors}
        \textbf{Modern Internet services}, such as search, social networking, and e-commerce, \textbf{critically depend on high-performance key-value stores}. Rendering even a single web page often requires hundreds or even thousands of storage accesses.
    \end{aquote}

    \vspace{7mm}

    % Need to scale up
    \begin{aquote}{SHArP\footnotemark{} authors}
        As the number of compute elements grows, and the need to expose and utilize higher levels of parallelism grows, \textbf{it is essential to [...] focus on developing architectures that lend themselves better to providing extreme-scale simulation capabilities}.
    \end{aquote}

    \setcounter{footnote}{1}
    \footnotetext{\cite{netchain}, \textsuperscript{2}\cite{sharp}}
\end{frame}

\glsreset{inp}
\begin{frame}[fragile]{\gls*{inp}}
    \inputifexists{sections/introduction/beamernotes/inp.tex}

    \setcounter{footnote}{0}

    \begin{itemize}
        \item INP refers to the technique of \textbf{offloading parts of the computation to network devices} (e.g., programmable switches, network accelerators, middleboxes, etc.), hence reducing the load on servers
        \item Advantages:
        \begin{enumerate} % From IncBricks
            \item Serve network requests on the fly with low latency
            \item Reduce data center traffic and mitigate network congestion
            \item Save energy by running servers in a low-power mode
        \end{enumerate}
        % \item (Daiet introduction: "The functionality of networks can now be enriched without hardware changes while retaining the capability of processing packets at very high rates, even above Terabits per second") \cite{daiet}
        \item Few solutions out there already: Daiet\footnotemark{}, SHArP\footnotemark{}, NetChain\footnotemark{}, IncBricks\footnotemark{}
    \end{itemize}

    \setcounter{footnote}{1}
    \footnotetext{\cite{daiet}, \textsuperscript{2}\cite{sharp}, \textsuperscript{3}\cite{netchain}, \textsuperscript{4}\cite{incbricks}}
\end{frame}

\begin{frame}[fragile]{Thesis goals}
    \inputifexists{sections/introduction/beamernotes/goals.tex}
    
    \textbf{Problem statement}\\
    For the time being, it seems that there is still no \gls*{rm} that takes into account the presence of a network having a data plane that supports (partially or completely) INP

    \vspace{5mm}

    \textbf{Goals}\\
    \begin{enumerate}
        \item Model and evaluate an API through which applications can ask for INP resources
        \item Discuss the importance of a scheduler which can reject INP requests and propose their server-only equivalent when needed (e.g., high switch utilization)
    \end{enumerate}

\end{frame}
