% Commands
% !TEX encoding = UTF-8 Unicode
% !TEX TS-program = pdflatex

% Document class
\documentclass[%
    corpo=12pt,
    twoside,
    %    stile=classica,
    oldstyle,
    %    autoretitolo,
    tipotesi=magistrale,
    greek,
    evenboxes,
]{toptesi}
\usepackage[utf8]{inputenc}
\usepackage[T1]{fontenc} % font encoding
\usepackage{lmodern} % font
\pdfminorversion=5

% Table of contents
\usepackage[nottoc,notlof,notlot]{tocbibind} % including references
\setcounter{tocdepth}{2} % depth
\setcounter{secnumdepth}{3} % section numbering depth

% Front page
\renewcommand\IDlabel{\\\quad Student ID:\xspace}

% Miscellaneous
% \usepackage{fullpage} % different margins
\usepackage[hidelinks]{hyperref} % links
% \hypersetup{
%     pdfpagemode={UseOutlines},
%     bookmarksopen,
%     pdfstartview={FitH},
%     colorlinks,
%     linkcolor={blue},
%     citecolor={blue},
%     urlcolor={blue}
% }
\usepackage[binary-units=true]{siunitx} % MB, GB, etc.
\usepackage{pdfpages} % import PDFs
\usepackage{verbatim} % multi-line comments

% Checkmarks and x-marks
\usepackage{amssymb}
\usepackage{pifont}
\definecolor{green}{RGB}{0,176,80}
\newcommand{\cmark}{{\color{green}\textbf{\ding{51}}} }
\definecolor{red}{RGB}{222,0,0}
\newcommand{\xmark}{{\color{red}\textbf{\ding{55}}} }

% Adding a dot at the end of paragraph titles
\let\originalparagraph\paragraph
\renewcommand{\paragraph}[2][.]{\originalparagraph{#2#1}}

% Glossaries and acronyms
\usepackage[acronym,toc]{glossaries} % package
% General
\newacronym{api}{API}{Application Programming Interface}
\newacronym{rm}{RM}{Resource Manager}
\newacronym{vm}{VM}{Virtual Machine}
\newacronym{inp}{INP}{In-Network Processing}
\newacronym{nfv}{NFV}{Network Function Virtualization}
\newacronym{sdn}{SDN}{Software Defined Networking}
\newacronym{voc}{VOC}{Virtual Oversubscribed Cluster}
\newacronym{tor}{ToR}{Top of Rack}
\newacronym{mpi}{MPI}{Message Passing Interface}
\newacronym{hpc}{HPC}{High Performance Computing}

% CloudMirror
\newacronym{tag}{TAG}{Tenant Application Graph}

% SHArP
\newacronym{an}{AN}{Aggregation Node}
\newacronym{am}{AM}{Aggregation Manager}
\newacronym{tca}{TCA}{Target Channel Adapter}
\newacronym{qp}{QP}{Queue Pair} % acronyms
\paragraph{Data center architecture} \label{dc_architecture}
% Data center fat-tree
    % Also network resources
        % Uniform compute units (CU) and properties
This simulator emulates \glspl{resource:physical:switch} besides \glslink{resource:physical:server}{server ones}.
It can support any kind of data center network architecture.
Specifically, a \textit{fat-tree} has been used for this evaluation, which is a very common topology for data centers to use. An example of a fat-tree topology is depicted in \autoref{fig:fattree}.
\begin{figure}[!htb]
    \centering
    \usebox{\fattree}
    \caption{A fat-tree topology with 4 \textit{pods}}
    \label{fig:fattree}
\end{figure}

This fat-tree has three layers of \glslink{resource:physical:switch}{switches}: \textit{core}, \textit{aggregation} and the last one which is usually called \textit{edge}, \textit{layer}, \textit{access} or just simply \gls{tor} \glslink{resource:physical:switch}{switches}.
Being $k$ the number of \textit{pods} in the topology, a fat-tree contains $(k/2)^2$ core \glslink{resource:physical:switch}{switches}, $k^2/2$ aggregation \glslink{resource:physical:switch}{switches}, $k^2/2$ \gls{tor} \glslink{resource:physical:switch}{switches}, and supports up to $k^3/4$ \glslink{resource:physical:server}{servers}.
Being then $5k^2/4$ the total amount of \glslink{resource:physical:switch}{switches} in a fat-tree, \glslink{resource:physical:server}{servers} become more abundant when $k>5$.

\paragraph{\Glspl{resource:physical:switch}} \label{simulator_switch_resources}
\glsreset{cu}
For the sake of simplicity, \glslink{resource:physical:switch}{physical switches} have a single numerical dimension called \gls{cu}.
This assumption does not affect this evaluation's results and it makes the scheduling algorithm a bit simpler.
Increasing the number of \glslink{resource:physical:switch}{switch} dimensions is trivial since the simulator already supports multiple dimensions for \glslink{resource:physical:server}{servers}, namely CPU and memory.
\Glspl{resource:logical:edge} have been ignored to simplify the placement algorithm.

\glslink{resource:physical:switch}{Switches} are also characterized by a \textit{property map}, that ultimately suggests which kinds of \gls{inp} solutions they are able to run.
In this evaluation, properties and \gls{inp} solutions coincide (e.g., \glslink{resource:physical:switch}{switch} $A$ supports \gls{inp} solutions $X$, $Y$, and $Z$), but these properties might be more appropriately extended to any other kind of hardware property that distinguish \glslink{resource:physical:switch}{switches} in the data center (e.g., CPU architecture, supported data plane programming language, etc.).
In general, a switch task requesting for property $P$ will only be allocated on a \glslink{resource:physical:switch}{switch} if it supports that same property $P$. % resources glossary
\makenoidxglossaries % computing the glossary and acronyms list

% In-line lists
\usepackage[inline]{enumitem}
\newlist{mylist}{enumerate*}{1}
\setlist[mylist]{label=(\roman*)}

% References sections
\renewcommand{\sectionautorefname}{\S}
\renewcommand{\subsectionautorefname}{\S}
\renewcommand{\subsubsectionautorefname}{\S}
\renewcommand\bibname{References}

% Capitalized references
\usepackage{cleveref}

% Footnotes with symbols
\usepackage[symbol]{footmisc}
\renewcommand{\thefootnote}{\fnsymbol{footnote}}

% Blank pages
\usepackage{afterpage}
\newcommand\blankpage{%
    \null
    \thispagestyle{empty}%
    \addtocounter{page}{-1}%
    \newpage
}

% Comments
\newcommand\pnote[1]{\textit{\textcolor{blue}{[p@]: #1}}}
\newcommand\ma[1]{\textit{\textcolor{purple}{[marcel]: #1}}}
\newcommand\lin[1]{\textit{\textcolor{green!55!blue}{[lin]: #1}}}
\newcommand\marco[1]{\textit{\textcolor{red}{[marco]: #1}}}
\newcommand\fulvio[1]{\textit{\textcolor{orange}{[fulvio]: #1}}}

\title{Data center resource management for in-network processing}
% \subtitle{A modern beamer theme}
\date{} % FIXME
\author{Marco Micera}
\institute{Politecnico di Torino, Technische Universit{\"a}t Darmstadt}
% \titlegraphic{\hfill\includegraphics[height=1.5cm]{logo.pdf}}

\begin{document}

\maketitle

\begin{frame}{Outline}
  \setbeamertemplate{section in toc}[sections numbered]
  \tableofcontents[hideallsubsections]
\end{frame}

\section{Introduction}

\begin{frame}[fragile]{Introduction}
  \begin{itemize}
    \item (NetCache introduction: "modern Internet services, such as search, social networking and e-commerce, critically depend on high-performance key-value stores. Rendering even a single web page often requires hundreds or even thousands of storage accesses." \cite{netchain}
    \item need to scale up (SHArP introduction, \textit{justifying INP}: "As the number of compute elements grows, and the need to expose and utilize higher levels of parallelism grows, it is essential to reconsider system architectures, and focus on developing architectures that lend themselves better to providing extreme-scale simulation capabilities.") \cite{sharp}
    \item However, modern-day data centers only exploit servers to perform computation % FIXME
  \end{itemize}

\end{frame}
\begin{frame}[fragile]{In-Network Processing (INP)}
  \begin{itemize}
    
    \item Offloading computation to network devices (e.g., programmable switches, network accelerators, middleboxes, etc.), hence reducing load on servers
    \item (Daiet introduction: "The functionality of networks can now be enriched without hardware changes while retaining the capability of processing packets at very high rates, even above Terabits per second") \cite{daiet}
    \item Few solutions out there already: Daiet \cite{daiet}, SHArP \cite{sharp}, NetChain \cite{netchain}, IncBricks \cite{incbricks}
  \end{itemize}
\end{frame}
\begin{frame}[fragile]{Problem statement}
  \begin{itemize}
    \item there is no \gls{rm} that considering server and switch resources \textbf{conjunctly}...
    % TODO mention network-aware RMs (verbally?)
  \end{itemize}
\end{frame}
\begin{frame}[fragile]{Goals}
  (abstract)
  \begin{enumerate}
    \item Model and evaluate an API through which applications can ask for INP resources
    \item Discuss the importance of a scheduler which can reject INP requests and propose their server-only equivalent when needed (e.g., high switch utilization)
  \end{enumerate}
\end{frame}

\section{Analysis}

\begin{frame}{Currently existing \gls{inp} solutions}
  \begin{itemize}
    \item In-network \textbf{storage}
    \begin{itemize}
      \item Switches must
      \begin{itemize}
        \item dedicate part of their local memory to store a distributed map
        \item form a chain
      \end{itemize}
      \item IncBricks \cite{incbricks}, NetChain \cite{netchain}
    \end{itemize}
		\item In-network \textbf{data aggregation}
    \begin{itemize}
      \item Switches must
      \begin{itemize}
        \item form a tree whose root is connected to data consumers and whose leaves are connected to data producers
        \item dedicate part of their local memory to store a key-value map
        \item be able to perform basic operations on data, such as writing and hashing
        \item wait for all its children to send aggregated data
      \end{itemize}
      \item Daiet \cite{daiet}, SHArP \cite{sharp}
    \end{itemize}
	\end{itemize}
\end{frame}
\begin{frame}{In-network caching system: IncBricks}
  % \begin{itemize}
  %   \item Custom fabric \textit{IncBox} (programmable switch + network accelerator)
  % \end{itemize}
  \centering
  \includegraphics[page=1, clip, trim=3.6cm 0.7cm 2.5cm 3.75cm, width=\textwidth]{analysis/inp/solutions.pdf}
\end{frame}
\begin{frame}{In-network caching system: IncBricks}
  \centering
  \includegraphics[page=2, clip, trim=3.6cm 0.7cm 2.5cm 3.75cm, width=\textwidth]{analysis/inp/solutions.pdf}
\end{frame}
\begin{frame}{In-network caching system: IncBricks}
  \centering
  \includegraphics[page=3, clip, trim=3.6cm 0.7cm 2.5cm 3.75cm, width=\textwidth]{analysis/inp/solutions.pdf}
\end{frame}
\begin{frame}{Coordination services: NetChain}
  \centering
  \includegraphics[page=6, clip, trim=3.6cm 0.7cm 3.2cm 4cm, width=\textwidth]{analysis/inp/solutions.pdf}
\end{frame}
\begin{frame}{In-network aggregation: Daiet}
  % Daiet \cite{daiet} is a system that performs in-network data aggregation for partition\hyp{}aggregate data center applications (big data analysis such as MapReduce \cite{mapreduce}, machine learning, graph processing, and stream processing). Inventors claim to achieve an 86.9\%-89.3\% traffic reduction, causing the execution time at the reducer to drop by 83.6\% on average.
  \centering
  \includegraphics[page=10, clip, trim=0.5cm 0.7cm 1.2cm 2.6cm, width=\textwidth]{analysis/inp/solutions.pdf}
\end{frame}
\begin{frame}{In-network aggregation: Daiet}
  \centering
  \includegraphics[page=11, clip, trim=0.35cm 0.6cm 0.3cm 2.6cm, width=\textwidth]{analysis/inp/solutions.pdf}
\end{frame}
\begin{frame}{Aggregation protocol: SHArP}
  \centering
  \includegraphics[page=15, clip, trim=3.3cm 0.9cm 1.6cm 3.4cm, width=\textwidth]{analysis/inp/solutions.pdf}
\end{frame}
\begin{frame}{Resource models (3.4)}
  \begin{columns}[T,onlytextwidth]
    \column{0.5\textwidth}
      % Items
      \begin{itemize}
        \item \glsentryfull{vc}\\
        \begin{center}
          \includegraphics[page=1, clip, trim=0.5cm 0.6cm 0.5cm 1cm, width=0.7\textwidth]{analysis/models/solutions.pdf}
        \end{center}
        \item \glsentryfull{voc}\\
        \begin{center}
          \includegraphics[page=2, clip, trim=0.5cm 0.6cm 0.5cm 0.7cm, width=\textwidth]{analysis/models/solutions.pdf}
        \end{center}
      \end{itemize}

    \column{0.5\textwidth}
      \begin{itemize}
        \item \glsentryfull{tag}\\
        \begin{center}
          \includegraphics[page=3, clip, trim=0.5cm 0.5cm 0.5cm 0.5cm, width=0.7\textwidth]{analysis/models/solutions.pdf}
        \end{center}
        \item Fine-grained resource requests
        \item High-level goals
      \end{itemize}
  \end{columns}
\end{frame}
\begin{frame}{Intregrating INP resources in RMs}
  (3.3)
\end{frame}
\begin{frame}{FIXME: the only “network-aware” RM + its problems}
  (3.3.2)
\end{frame}

\section{FIXME Requirements? It is necessary?}

\section{Design}
\begin{frame}{Composites}
  (Thesis repository \texttt{dock/thesis/figures/design/model/presentation.pdf})
\end{frame}
\begin{frame}{The extended-Tenant Application Graph (eTag)}
  \begin{itemize}
    \item (5.2.1 why existing resource models do not satisfy all requirements)
    \item (5.2.2 eTag)
  \end{itemize}
\end{frame}
\begin{frame}{The template database}
  \begin{itemize}
    \item (5.3 generic groups)
    \item (5.1.2 template database role)(5.1.2 template database role)
  \end{itemize}
\end{frame}
\begin{frame}{FIXME I should introduce composites earlier}
  % TODO I should mention generic groups
\end{frame}

\begin{frame}{Bibliography}
  Some references to showcase [allowframebreaks] \cite{netchain, incbricks, daiet, sharp}
\end{frame}

\section{Conclusions}

% \begin{frame}{Summary}

%   Get the source of this theme and the demo presentation from

%   \begin{center}\url{github.com/matze/mtheme}\end{center}

%   The theme \emph{itself} is licensed under a
%   \href{http://creativecommons.org/licenses/by-sa/4.0/}{Creative Commons
%   Attribution-ShareAlike 4.0 International License}.

%   \begin{center}\ccbysa\end{center}

% \end{frame}

\begin{frame}[standout]
  Questions?
\end{frame}

\begin{frame}[standout]
  Thank you
\end{frame}

\appendix

% \begin{frame}[fragile]{Backup slides}
%   Sometimes, it is useful to add slides at the end of your presentation to
%   refer to during audience questions.

%   The best way to do this is to include the \verb|appendixnumberbeamer|
%   package in your preamble and call \verb|\appendix| before your backup slides.

%   \themename will automatically turn off slide numbering and progress bars for
%   slides in the appendix.
% \end{frame}

\begin{frame}[allowframebreaks]{Bibliography}
  \bibliographystyle{abbrv}
  \bibliography{../thesis/utility/references}
\end{frame}

\end{document}
