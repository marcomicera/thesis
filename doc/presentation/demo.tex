% Commands
% Document class
\documentclass[a4paper, 11pt]{report}
\usepackage[utf8]{inputenc}

% Table of contents
\usepackage[nottoc,notlof,notlot]{tocbibind} % including references
\setcounter{tocdepth}{2} % depth
\setcounter{secnumdepth}{3} % section numbering depth

% Miscellaneous
\usepackage{fullpage} % different margins
\usepackage{hyperref} % links
\usepackage[binary-units=true]{siunitx} % MB, GB, etc.
\usepackage{pdfpages} % import PDFs
\usepackage{verbatim} % multi-line comments

% Checkmarks and xmarks
\usepackage{amssymb}
\usepackage{pifont}
\definecolor{green}{RGB}{0,176,80}
\newcommand{\cmark}{{\color{green}\textbf{\ding{51}}} }
\definecolor{red}{RGB}{222,0,0}
\newcommand{\xmark}{{\color{red}\textbf{\ding{55}}} }

% Adding a dot at the end of paragraph titles
\let\originalparagraph\paragraph
\renewcommand{\paragraph}[2][.]{\originalparagraph{#2#1}}

% Glossaries and acronyms
% section,numberedsection=autolabel
\usepackage[acronym,toc]{glossaries} % package
% General
\newacronym{rm}{RM}{Resource Manager}
\newacronym{rmf}{RMF}{Resource Management Framework}
\newacronym{vm}{VM}{Virtual Machine}
\newacronym{inp}{INP}{In-Network Processing}
\newacronym{nfv}{NFV}{Network Function Virtualization}
\newacronym{sdn}{SDN}{Software Defined Networking}
\newacronym{tor}{ToR}{Top of Rack}
\newacronym{hpc}{HPC}{High Performance Computing}
\newacronym{dht}{DHT}{Distributed hash table}

% Programming
\newacronym{api}{API}{Application Programming Interface}
\newacronym{mpi}{MPI}{Message Passing Interface}
\newacronym{rpc}{RPC}{Remote Procedure Call}

% Resource models
\newacronym{vc}{VC}{Virtual Cluster}
\newacronym{voc}{VOC}{Virtual Oversubscribed Cluster}
\newacronym{tivc}{TIVC}{Time-Interleaved Virtual Cluster}
\newacronym{tag}{TAG}{Tenant Application Graph}

% SHArP
\newacronym{an}{AN}{Aggregation Node}
\newacronym{am}{AM}{Aggregation Manager}
\newacronym{tca}{TCA}{Target Channel Adapter}
\newacronym{qp}{QP}{Queue Pair}
\newglossaryentry{ibm}{
    name=IBM,
    description=IBM\textsuperscript{\textregistered}
}
\newglossaryentry{switchib2}{
    name=Mellanox's SwitchIB-2,
    description=Mellanox's SwitchIB-2\textsuperscript{TM}
}

% Apache Hadoop YARN
\newglossaryentry{apache}{
    name=Apache,
    description=Apache\textsuperscript{TM}
}
\newglossaryentry{hadoop}{
    name=Apache Hadoop,
    description=\glsdesc{apache} Hadoop\textsuperscript{\textcopyright}
}
\newglossaryentry{yarn_full}{
    name=Apache Hadoop YARN,
    description=\glsdesc{hadoop} YARN \texorpdfstring{\cite{yarn}}{}
}
\newglossaryentry{yarn}{
    name=Apache YARN,
    description=\glsdesc{apache} YARN \texorpdfstring{\cite{yarn}}{}
} % acronyms
% Glossary definition
\newglossary*{resources}{Resources glossary}

\newglossaryentry{resource:physical}{
    type=resources,
    name=Physical resource,
    text=physical resource,
    description={physical hardware component of limited availability within a physical machine}
}

    \newglossaryentry{resource:physical:server}{
        type=resources,
        name=Physical server resource,
        text=physical server resource,
        parent=resource:physical,
        description={resource of a physical server machine}
    }
    
    \newglossaryentry{resource:physical:switch}{
        type=resources,
        name=Physical switch resources,
        text=physical switch resources,
        parent=resource:physical,
        description={resource of a physical switche, network accelerator, middle-box and of every kind of network device originally intended to forward packets}
    }
    
\newglossaryentry{resource:logical}{
    type=resources,
    name=Logical resource,
    text=logical resource,
    description={logical representation of a physical resource}
}

    \newglossaryentry{resource:logical:server}{
        type=resources,
        name=Logical server resource,
        text=logical server resource,
        parent=resource:logical,
        description={virtualized server physical resource, often implemented by means of a \gls{vm}, container or an entire physical server}
    }
    
    \newglossaryentry{resource:logical:switch}{
        type=resources,
        name=Logical switch resource,
        text=logical switch resource,
        parent=resource:logical,
        description={logical representation of a physical switch resource not mapped to any physical switch device}
    }
    
    \newglossaryentry{resource:logical:edge}{
        type=resources,
        name=Logical edge resource,
        text=logical edge resource,
        parent=resource:logical,
        description={properties of virtual connections between two logical resources, e.g., bandwidth, latency, etc}
    }
    
\newglossaryentry{resource:composite}{
    type=resources,
    name=Composite,
    text=composite,
    description={template describing a high-level logical component.
        It can be made out of other composites and/or logical resources.}
}

    \newglossaryentry{resource:composite:server}{
        type=resources,
        name=Server composite,
        text=server composite,
        parent=resource:composite,
        description={composite describing a high-level server component, e.g., \textit{web server}, \textit{databases}, etc}
    }
    
    \newglossaryentry{resource:composite:inp}{
        type=resources,
        name=\gls{inp} composite,
        text=\texorpdfstring{\gls{inp}}{INP} composite,
        parent=resource:composite,
        description={composite describing a high-level \gls{inp} application, e.g., \textit{IncBricks} \cite{incbricks}, \textit{NetChain} \cite{netchain}, etc}
    }
    
\newglossaryentry{model}{
    type=resources,
    name=Resource model,
    text=resource model,
    description={model capable of describing composites and logical resources.
        The model exposed to tenants and the one internally used by the \gls{rm} could be different}
}

    \newglossaryentry{model:tenant}{
        type=resources,
        name=Tenant-side model,
        text=tenant-side model,
        parent=model,
        description={resource model exposed to tenants by the system \gls{api}}
    }
    
    \newglossaryentry{model:rm}{
        type=resources,
        name=\gls{rm}-side model,
        text=\texorpdfstring{\gls{rm}}{RM}-side model,
        parent=model,
        description={resource model internally used by the placement algorithm in order to allocate logical resources}
    } % resources glossary
\makeglossaries % computing the glossary and acronyms list

% In-line lists
\usepackage[inline]{enumitem}
\newlist{mylist}{enumerate*}{1}
\setlist[mylist]{label=(\roman*)}

% References sections
\renewcommand{\sectionautorefname}{\S}
\renewcommand{\subsectionautorefname}{\S}
\renewcommand{\subsubsectionautorefname}{\S}
\renewcommand\bibname{References}

% Capitalized references
\usepackage{cleveref}

% Footnotes with symbols
\usepackage[symbol]{footmisc}
\renewcommand{\thefootnote}{\fnsymbol{footnote}}

\title{Data center resource management for in-network processing}
% \subtitle{A modern beamer theme}
% \date{March 26th, 2020} % FIXME
\author{Marco Micera}
% \institute{Center for modern beamer themes}
% \titlegraphic{\hfill\includegraphics[height=1.5cm]{logo.pdf}}

\begin{document}

\maketitle

\begin{frame}{Table of contents}
  \setbeamertemplate{section in toc}[sections numbered]
  \tableofcontents[hideallsubsections]
\end{frame}

\section{Introduction}

\begin{frame}[fragile]{\glsentryfull{inp}}
  (2.4)
\end{frame}
\begin{frame}[fragile]{Problem statement}
  ...RMs do not consider server and switch resources \textbf{conjunctly}...
\end{frame}
\begin{frame}[fragile]{Goals}
  (abstract)
  \begin{enumerate}
    \item Model and evaluate an API through which applications can ask for INP resources
    \item Discuss the importance of a scheduler which can reject INP requests and propose their server-only equivalent when needed (e.g., high switch utilization)
  \end{enumerate}
\end{frame}

\section{Analysis}

\begin{frame}{Currently existing \gls{inp} solutions}
	\begin{itemize}
		\item In-network data aggregation
    \begin{itemize}
      \item Properties (5.3.1)…
      \item Daiet, SHArP
    \end{itemize}
		\item In-network storage
    \begin{itemize}
      \item Properties (5.3.1)…
      \item NetChain, IncBricks
    \end{itemize}
	\end{itemize}
\end{frame}
\begin{frame}{In-network aggregation: Daiet}
\end{frame}
\begin{frame}{Aggregation protocol: SHArP}
\end{frame}
\begin{frame}{Coordination services: NetChain}
\end{frame}
\begin{frame}{In-network caching [fabric]: IncBricks}
\end{frame}
\begin{frame}{Resource models (3.4)}
  \begin{columns}[T,onlytextwidth]
    \column{0.5\textwidth}
      % Items
      \begin{itemize}
        \item VC
        \begin{itemize}
          \item Image
        \end{itemize}
        \item VOC
        \begin{itemize}
          \item Image
        \end{itemize}
        \item TAG
        \begin{itemize}
          \item Image
        \end{itemize}
      \end{itemize}

    \column{0.5\textwidth}
      % Enumerations
      \begin{enumerate}
        \item Fine-grained resource requests
        \item High-level goals
      \end{enumerate}
  \end{columns}
\end{frame}
\begin{frame}{Intregrating INP resources in RMs}
  (3.3)
\end{frame}
\begin{frame}{FIXME: the only “network-aware” RM + its problems}
  (3.3.2)
\end{frame}

\section{FIXME Requirements? It is necessary?}

\section{Design}
\begin{frame}{Composites}
  (Thesis repository \texttt{dock/thesis/figures/design/model/presentation.pdf})
\end{frame}
\begin{frame}{The extended-Tenant Application Graph (eTag)}
  \begin{itemize}
    \item (5.2.1 why existing resource models do not satisfy all requirements)
    \item (5.2.2 eTag)
  \end{itemize}
\end{frame}
\begin{frame}{The template database}
  \begin{itemize}
    \item (5.3 generic groups)
    \item (5.1.2 template database role)(5.1.2 template database role)
  \end{itemize}
\end{frame}
\begin{frame}{FIXME I should introduce composites earlier}
\end{frame}

\begin{frame}{Bibliography}
  Some references to showcase [allowframebreaks] \cite{netchain, incbricks, daiet, sharp}
\end{frame}

\section{Conclusions}

% \begin{frame}{Summary}

%   Get the source of this theme and the demo presentation from

%   \begin{center}\url{github.com/matze/mtheme}\end{center}

%   The theme \emph{itself} is licensed under a
%   \href{http://creativecommons.org/licenses/by-sa/4.0/}{Creative Commons
%   Attribution-ShareAlike 4.0 International License}.

%   \begin{center}\ccbysa\end{center}

% \end{frame}

\begin{frame}[standout]
  Questions?
\end{frame}

\begin{frame}[standout]
  Thank you
\end{frame}

\appendix

% \begin{frame}[fragile]{Backup slides}
%   Sometimes, it is useful to add slides at the end of your presentation to
%   refer to during audience questions.

%   The best way to do this is to include the \verb|appendixnumberbeamer|
%   package in your preamble and call \verb|\appendix| before your backup slides.

%   \themename will automatically turn off slide numbering and progress bars for
%   slides in the appendix.
% \end{frame}

\begin{frame}[allowframebreaks]{Bibliography}
  \bibliographystyle{abbrv}
  \bibliography{../thesis/utility/references}
\end{frame}

\end{document}
