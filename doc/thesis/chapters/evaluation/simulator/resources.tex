\paragraph{Data center architecture}
% Data center fat-tree
    % Also network resources
        % Uniform compute units (CU) and properties
This simulator emulates \glspl{resource:physical:switch} besides \glslink{resource:physical:server}{server ones}.
The simulator can support any kind of data center network architecture, but a \textit{fat-tree} has been used for this evaluation, which is a very common topology for data centers to use.
% TODO fat-tree picture?

This fat-tree has three layers of \glslink{resource:physical:switch}{switches}: \textit{core}, \textit{aggregation} and a last one which is usually called \textit{edge}, \textit{layer}, \textit{access} or just simply \gls{tor} \glslink{resource:physical:switch}{switches}.
Being $k$ the number of \textit{pods} in the topology, a fat-tree contains $(k/2)^2$ core \glslink{resource:physical:switch}{switches}, $k^2/2$ aggregation \glslink{resource:physical:switch}{switches}, $k^2/2$ \gls{tor} \glslink{resource:physical:switch}{switches}, and supports up to $k^3/4$ \glslink{resource:physical:server}{servers}.

\paragraph{\Glspl{resource:physical:switch}} \label{simulator_switch_resources}
\glsreset{cu}
For the sake of simplicity, \glslink{resource:physical:switch}{physical switches} have a single numerical dimension called \gls{cu}.
This assumption does not affect this evaluation's results and it makes the scheduling algorithm a bit simpler.
Increasing the number of \glslink{resource:physical:switch}{switch} dimensions is trivial since the simulator already supports multiple dimensions for \glslink{resource:physical:server}{servers}, namely CPU and memory.
\Glspl{resource:logical:edge} have been ignored to simplify the placement algorithm.

\glslink{resource:physical:switch}{Switches} are also characterized by a \textit{property map}, that ultimately suggests which kinds of \gls{inp} solutions they're able to run.
In this evaluation, properties and \gls{inp} solutions coincide (e.g., \glslink{resource:physical:switch}{switch} $A$ supports \gls{inp} solutions $X$, $Y$ and $Z$), but these properties might be more appropriately extended to any other kind of hardware property that distinguish that \glslink{resource:physical:switch}{switch} in the data center (e.g., CPU architecture, supported data plane programming language, etc.).
In general, a switch task requesting for property $P$, will only be allocated on a \glslink{resource:physical:switch}{switch} if it supports that property $P$.