Nowadays there exist several \gls{inp} solutions that allow tenants to improve their application performance in terms of different metrics: \textsc{Daiet} \cite{daiet} inventors claim to achieve an 86.9\%-89.3\% traffic reduction by performing data aggregation entirely in the network data plane.
Other solutions like {NetChain \cite{netchain} and IncBricks \cite{incbricks} let programmable switches store data and process queries to cut end-to-end latency;
CloudMirror \cite{cloudmirror} allows client applications to specify bandwidth and high availability guarantees.

For the time being, it seems that there is still no valid resource allocation algorithm that takes into account the presence of a network having a data plane that supports (partially or completely) \gls{inp}.
This thesis has mainly two goals:
\begin{mylist}
    \item model and evaluate an \gls{api} through which applications can ask for \gls{inp} resources and % while providing guarantees (e.g., bandwidth). 
    \item discuss the importance of a scheduler which can reject \gls{inp} requests and propose their server-only equivalent when needed (e.g., high switch utilization) 
\end{mylist}.