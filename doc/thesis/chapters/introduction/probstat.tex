Using \gls{inp} to scale up data centers and improve their overall performance seems a promising idea: Daiet \cite{daiet} inventors claim to achieve an 86.9\%-89.3\% traffic reduction, hence reducing workload at the servers;
NetChain \cite{netchain} can process queries entirely in the network data plane, thus eliminating the query processing at servers and cutting the end-to-end latency to as little as half of an \glsentryshort{rtt}.\\
Current data center \glspl{rm} (such as Apache YARN \cite{yarn}, Google Omega \cite{omega}) are not completely network-unaware: for instance, some of them are capable of satisfying affinity rules.
CloudMirror \cite{cloudmirror} even provides bandwidth guarantees to tenant applications.
Still, current \glspl{rm} do not consider \gls{inp} resources.
As a consequence, tenant applications cannot request these kind of services while asking for server resources.

\subsection{Modeling \texorpdfstring{\glsentryshort{inp}}{INP} resources}
One of the two goals of this Master's thesis consists in investigating how to model \gls{inp} resources and how to integrate them in RMs.
In order to offer \gls{inp} services to a tenant application, the latter should be able to ask for \gls{inp} resources through an \gls{api}.
To do that, \gls{inp} resources must be modeled not only to support currently existing \gls{inp} solutions such as \cite{daiet} \cite{netchain} \cite{incbricks} \cite{sharp}, but also future ones. 
It may also be convenient to derive a single model to describe both server and \gls{inp} resources.

Classic tenant application requests can often be modeled as a key-value data structure.
CloudMirror \cite{cloudmirror} requires a \gls{tag} as an input, which is a directed graph where each vertex represents an application component and links' weights represent the minimum requested bandwidth.
One possible model could be based on a \gls{tag}, describing network resources and \gls{inp} services as vertexes or links.
Tenant applications could either use the same model used within the data center or a simplified one, adding another level of abstraction.

\subsection{\texorpdfstring{\glsentryshort{inp}}{INP}-aware \texorpdfstring{\glsentrylongpl{rm}}{Resource Managers}} \label{inp_aware_rms}
\glsreset{rm}
In order for everything to work, a network-aware placement algorithm in the \glsentrylong{rm} should be able to consider \gls{inp} and \glslink{resource:logical:server}{server} resources conjunctly: this brings new challenges in the field of resource management as there are currently no \glspl{rm} doing this.
One problem that could arise is due to the fact that \gls{inp} resources are typically very limited in a data center: \autoref{conclusions} will argue the importance of an \gls{rm} which is flexible enough to propose alternatives based on the current utilization of \gls{inp} and \glslink{resource:logical:server}{server} resources, since one kind of \glslink{resource:physical}{physical resource} type can become the bottleneck for the other.