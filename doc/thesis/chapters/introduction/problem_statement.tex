Using INP to keep scaling data centers' performance seems a promising idea: Daiet \cite{daiet} inventors claim to achieve a 86.9\%-89.3\% traffic reduction, hence reducing servers' workload; NetChain \cite{netchain} can process queries entirely in the network data plane, eliminating the query processing at servers and cutting the end-to-end latency to as little as half of an RTT.\\
Current data center Resource Managers (RMs) (e.g., Apache YARN \cite{yarn}, Google Omega \cite{omega}) are not completely network-unaware: for instance, some of them are capable of satisfying affinity rules. CloudMirror \cite{cloudmirror} even provides bandwidth guarantees to tenant applicaitons. Still, current RMs do not consider INP resources.\\
As a consequence, tenant applications cannot request INP services while asking for server resources.

\subsection{Modeling INP resources}
This Master's thesis goal consists in investigating how to model INP resources and how to integrate them in RMs.\par
In order to offer INP services to a tenant application, the latter should be capable of asking for INP resources through an API. To do that, INP resources must be modeled not only to support currently existing INP solutions such as \cite{daiet} \cite{netchain} \cite{incbricks} \cite{sharp}, but also to support future ones. It might be convenient to derive a single model to describe both server and INP resources.\par
Classic tenant application requests can often be modeled as a key-value data structure. CloudMirror \cite{cloudmirror} requires a \textit{Tenant Application Graph} (TAG) as an input, which is a directed graph where each vertex represents an application component and links' weights represent the minimum requested bandwidth. One possible model could be based on a TAG, describing network resources and/or INP services as vertexes or links. Tenants applications could either use the same model used within the data center or a simplified one, adding another level of abstraction.\par
Finally, a network-aware placement algorithm in the Resource Manager should then be able to allocate the requested resources accordingly.