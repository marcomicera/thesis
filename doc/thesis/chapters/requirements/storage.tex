Examples:
\begin{itemize}
    \item NetChain \cite{netchain} for coordination services
    \item IncBricks \cite{incbricks} for in-network caching
\end{itemize}
Network devices must:
\begin{itemize}
    \item form a chain
    \begin{itemize}
        \item NetChain \cite{netchain}: the NetChain \cite{netchain} agent must explicitly specify the list of IP addresses of all switches belonging to the chain, since its order depends on the query type (read or write).
        \item IncBricks \cite{incbricks}: the chain must connect the two communicating nodes, having just one switch connected per node (the \gls{tor} switch).
    \end{itemize}
    \item dedicate part of their local memory to store a key-value map
    \begin{itemize}
        \item NetChain \cite{netchain}: distributed map, hash segments are repeated across multiple physical switches. Servers are never involved.
        \item IncBricks \cite{incbricks}: cache, zero or more switches can store the pair. Servers may be involved in case no switch has cached the pair.
    \end{itemize}
\end{itemize}
Data consumers:
\begin{itemize}
    \item execute queries
    \item are \glspl{vm}
\end{itemize}
Data producers:
\begin{itemize}
    \item own data
    \item are \glspl{vm}
    \begin{itemize}
        \item NetChain \cite{netchain}: are data consumers
        \item IncBricks \cite{incbricks}: are not data consumers
    \end{itemize}
\end{itemize}
The \gls{sdn} controller:
\begin{itemize}
    \item must be connected to all the network devices
    \begin{itemize}
        \item NetChain \cite{netchain}: must form the chain and handle switches reconfigurations
        \item IncBricks \cite{incbricks}: must configure network devices in order for them to forward IncBricks \cite{incbricks} packets accordingly
    \end{itemize}
\end{itemize}