\begin{itemize}
    \item \textbf{Physical resources}:
    hardware device resources.
    
    \begin{itemize}
        \item \textbf{Server physical resources}:
        
        \item \textbf{Switch physical resources} or \textbf{network devices}:
        physical switches, network accelerators, middle-boxes and every kind of device originally intended to forward packets.
        
        \item \textbf{Link physical resources}:
    \end{itemize}
    
    \item \textbf{Logical resources}:
    logical representation of physical resources.
    
    \begin{itemize}
        \item \textbf{Server logical resources}:
        virtualized server physical resources, often implemented with \glspl{vm}, containers or with entire server physical resources.
        
        \item \textbf{Switch logical resources}:
        
        \item \textbf{Edge logical resources}:
    \end{itemize}
    
    \item \textbf{Composites}:
    templates that describe high-level logical components.
    They can be made out of other composites and logical resources.
    
    \begin{itemize}
        \item \textbf{Server composites}:
        
        \item \textbf{\gls{inp} composites}:
    \end{itemize}
    
    \item \textbf{Resource model}:
    model capable of describing composites and logical resources.
    The model exposed to tenants and the one internally used by the \gls{rm} could be different.
    
    \begin{itemize}
        \item \textbf{Tenant-side model}:
        resource model exposed to tenants by the system \gls{api}.
        
        \item \textbf{\gls{rm}-side model}:
        resource model internally used by the placement algorithm in order to allocate logical resources.
    \end{itemize}
\end{itemize}