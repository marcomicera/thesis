\paragraph{Quick recap}
The \gls{voc} model is the oversubscribed extension of the \gls{vc} one.
\autoref{vocfigurerecap} shows an example.

\begin{figure}[!htb]
    \centering
    \usebox{\vocfigure}
    \caption{A graphical representation of the \gls{voc} model}
    \label{vocfigurerecap}
\end{figure}

\paragraph{Requirements check}
The \gls{voc} model has the advantage of not requiring a virtual switch of bandwidth $N \cdot B$, that could be unpractical when $N$ is particularly large.
When a tenant application issues a resource request using a \gls{voc}, it requests for a $<N, B, S, O>$ tuple, being respectively
\begin{mylist}
    \item the number of \glspl{vm},
    \item the bandwidth needed for intra-group communication,
    \item the group size and
    \item the oversubscription factor
\end{mylist}.
Still, certain kinds of applications might need different values of $B$ for different groups (\xmark \ref{requirements:model:bandwidth}).
Furthermore, the two-layer logical tree topology might not suit perfectly for applications that require a chain having more than 2 network devices or a tree topology with more than 2 layers (\xmark \ref{requirements:model:topology}).
The \gls{voc} model still does not support \gls{resource:logical:switch} requests (\xmark \ref{requirements:model:tenant:logical}).

\paragraph{A possible variant}
The two limitations previously mentioned can be obviously overcome by allowing tenants to specify
\begin{mylist}
    \item \glspl{resource:logical:switch} (satisfying \cmark \ref{requirements:model:tenant:logical})
    \item different bandwidth demands for different groups (satisfying \cmark \ref{requirements:model:bandwidth}) and
    \item an arbitrary tree height
\end{mylist}.
The corresponding variant is shown in \autoref{fig:vocmod}.

\begin{figure}[!htb]
    \centering
    \usebox{\vocmodfigure}
    \caption{A possible \gls{voc} variant with an arbitrary tree height}
    \label{fig:vocmod}
\end{figure}

However, tenant applications still
\begin{mylist}
    \item cannot express any kind of switch topology (e.g., a switch loop for an in-network \gls{dht} chord), violating \xmark \ref{requirements:model:topology}, and
    \item convert \gls{inp} high-level requirements (e.g., in-network total cache size, lock requests per second, etc.) into switch resource requirements (\xmark \ref{requirements:model:translation})
\end{mylist}.