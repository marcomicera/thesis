%\paragraph{Quick recap}
%The \gls{voc} model is the oversubscribed extension of the \gls{vc} one.
%
%\begin{figure}[!htb]
%    \centering
%    \usebox{\vocfigure}
%    \caption{A graphical representation of the \gls{voc} model}
%\end{figure}
%
\paragraph{Its role in \texorpdfstring{\gls{inp}}{INP}}
The \gls{voc} model has the advantage of not requiring a virtual switch of bandwidth $N \cdot B$, that could be unpractical when $N$ is particularly large.
When a tenant application issues a resource request using a \gls{voc}, it just requests for the $<N, B, S, O>$ tuple, being respectively
\begin{mylist}
    \item the number of \glspl{vm},
    \item the bandwidth needed for intra-group communication,
    \item the group size and
    \item the oversubscription factor
\end{mylist}.
Still, certain kinds of applications might need different values of $B$ for different groups.
Furthermore, the two-layer logical tree topology might not suit perfectly for applications that require a chain having more than 2 network devices or a tree topology with more than 2 layers.

\paragraph{A possible modification}
The two limitations previously mentioned can obviously be overcome by allowing to specify
\begin{mylist}
    \item a different bandwidth demand $B_i$ for each different group $i$ and
    \item an arbitrary tree height
\end{mylist}.
The corresponding variant is shown in \autoref{fig:vocmod}.

\begin{figure}[!htb]
    \centering
    \usebox{\vocmodfigure}
    \caption{A possible \gls{voc} variant with an arbitrary tree height}
    \label{fig:vocmod}
\end{figure}

There are currently two open problems in this variant:
\begin{mylist}
    \item the number of oversubscription factors to be specified grows exponentially with the tree height ($2^h-1$ factors with a tree of height $h$) and
    \item \textbf{\textit{TO BE CONTINUED}} ...
\end{mylist}.