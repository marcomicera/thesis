%\paragraph{Quick recap}
%The \gls{vc} model consists in $N$ \glspl{vm} connected to a single \textit{virtual switch} by a link of bandwidth $B$.
%
%\begin{figure}[!htb]
%    \centering
%    \usebox{\vcfigure}
%    \caption{A graphical representation of the \gls{vc} model}
%\end{figure}
%
\paragraph{Its role in \texorpdfstring{\gls{inp}}{INP}}
Expressing just one bandwidth value $B$ for all connections is obviously a big limitation which would cause a waste of bandwidth for those applications having nodes that require different amounts of bandwidth.
The \gls{tivc} model introduced in Proteus \cite{proteus} overcomes this limitation since bandwidth constraints are expressed by time-varying functions instead of a constant fixed value.
Both models though rely on a single virtual switch that is not suitable for tenant applications which require a more complex logical switch topology such as a chain or a tree.

\paragraph{A possible modification}
Since specifying the same bandwidth demand $B$ for all \glspl{vm} is restrictive, one trivial solution could consist in allowing the model to specify different values of $B$, like depicted in \autoref{fig:vcmod}.

\begin{figure}[!htb]
    \centering
    \usebox{\vcmodfigure}
    \caption{A possible \gls{vc} variant}
    \label{fig:vcmod}
\end{figure}

Still, the one-level tree topology remains a problem for those applications which require a more complex logical switch topology: regarding this, a natural \gls{vc} extension is the \gls{voc} model, hence leaving aside the former model in favor of the latter one.