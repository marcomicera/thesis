The proposed model is based on \gls{tag} and it allows tenant applications to specify
\begin{mylist}
    \item all \glspl{resource:composite} (\cmark \ref{requirements:model:tenant:composites}) and \glspl{resource:logical} (\cmark \ref{requirements:model:tenant:logical})
    \item different bandwidth demands (\cmark \ref{requirements:model:bandwidth}) for different entities
    \item any kind of network topology (\cmark \ref{requirements:model:topology})
\end{mylist}.
An example of proposed model is depicted in \autoref{tagmodfigure}.

\begin{figure}[!htb]
    \centering
    \usebox{\tagmodfigure}
    \caption{The proposed model based on \gls{tag}}
    \label{tagmodfigure}
\end{figure}

It is worth noticing that with the introduction of \glspl{resource:composite} in the model, the translation to \glspl{resource:logical} must be done by the \gls{rm} and not by the tenant application (\cmark \ref{requirements:model:translation}).