\glsreset{etag}
The \textit{\gls{etag}} is based on a \gls{tag} and it allows tenant applications to specify
\begin{mylist}
    \item all types of \glspl{resource:composite} (\cmark \ref{requirements:model:tenant:composites}) and \glspl{resource:logical} (\cmark \ref{requirements:model:tenant:logical}),
    \item different bandwidth demands (\cmark \ref{requirements:model:bandwidth}) for different entities and
    \item any kind of network topology (\cmark \ref{requirements:model:topology})
\end{mylist}.
An example of an \gls{etag} is depicted in \autoref{tagmodfigure}.

\begin{figure}[!htb]
    \centering
    \usebox{\tagmodfigure}
    \caption{The \gls{etag} proposed model}
    \label{tagmodfigure}
\end{figure}

The introduction of \glspl{resource:composite} in the model delegates their translation to \glspl{resource:logical} to the \gls{rm} and not by the tenant application (\cmark \ref{requirements:model:translation}).
The \textit{template database} (\autoref{system_design_overview}) allows tenants to express \gls{resource:composite:inp} requirements in terms of high-level properties (\cmark \ref{requirements:model:tenant:verbosity}).