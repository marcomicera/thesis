The \textit{template database}
\ifdefined\THESISSUMMARY \else
(\autoref{system_design_overview})
\fi
takes care of translating \glspl{resource:composite} into a set of \glspl{resource:logical} in order for the placement algorithm to
\ifdefined\THESISSUMMARY
work.
\else
work (\autoref{compositestological}).
\fi
Ideally, the template database should map individual \gls{inp} solutions to their equivalent made out of \glspl{resource:logical} only, but this may be not feasible as the number of \gls{inp} solutions may increase in the future.
That is to say, this approach is not scalable.
\ifdefined\THESISSUMMARY \else

\fi
\gls{inp} solutions may generally seem very different from each other as they pursue different goals, but grouping them based on common aspects might be a way to decrease the number of entries of a template database. % TODO explain this a bit better
That said, there is no unique way of categorizing \gls{inp} solutions: for instance, one could group them based on their final purpose, on their logical architecture, and so on.
This
\ifdefined\THESISSUMMARY
thesis
\else
section
\fi
tries to do this based on the latter
\ifdefined\THESISSUMMARY
aspect resulting in two different groups: in-network aggregation and in-network storage.
\else
aspect, using those \gls{inp} solutions described in \autoref{inp_solutions}.\fi
