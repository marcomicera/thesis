Two key examples of \gls{inp} solutions that might belong to such a group are Daiet \cite{daiet} and SHArP \cite{sharp}.
Usually \glslink{resource:physical:switch}{network devices} must
\begin{mylist}
    \item form a tree whose root is connected to data consumers and whose leaves are connected to data producers
    \item dedicate part of their local memory to store a key-value map
    \item be able to perform basic operations on data, such as writing and hashing
    \item wait for all its children to send aggregated data
\end{mylist}.
Then of course single \gls{inp} solutions may vary for different aspects: for instance, SHArP \cite{sharp} supports multiple data consumers, while Daiet \cite{daiet} expects only one.

Data producers must be connected to exactly one tree leaf, and data consumer(s) must be connected to the tree root.

Daiet \cite{daiet} and SHArP \cite{sharp} differ a lot when it comes to nodes management.
In Daiet \cite{daiet}, the \gls{sdn} controller must push flow rules to all switches belonging to at least one tree.
On the other hand, SHArP \cite{sharp} has a special unit (not necessarily be the \gls{sdn} controller) that acts as a sort of \gls{rm}, dedicating SHArP \cite{sharp} resources to those entities who request for them.