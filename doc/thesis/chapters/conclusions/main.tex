Based on the results achieved in \autoref{results} and on the analysis of network-aware \glspl{rm} in \autoref{rm_network_awereness}, this chapter aims to foresee some features that an ideal fully-\gls{inp} aware \glsentrylong{rm} should have, as well as some open problems in the \gls{inp} resource management field.

\section{Fully \texorpdfstring{\glsentryshort{inp}}{INP}-aware \texorpdfstring{\glsentryshort{rm}}{RM} features}

% "On Tackling" has the con of not placing server and switches conjunctly, during the same round
\paragraph{Conjunct placement}
As mentioned in \autoref{network_resources-aware_rms}, \cite{ontackling} considers both \glslink{resource:logical:switch}{server} and \glspl{resource:logical:switch} during allocation, but it does not place them during the same scheduling round.
The drawback of not scheduling these two kinds of \glslink{resource:logical}{resources} conjunctly causes the placement algorithm to derive a sub-optimal placement, as explained in \autoref{network_resources-aware_rms}.

Ideally, an \gls{rm} should allocate a job considering its \glslink{resource:logical:switch}{server} and \glspl{resource:logical:switch} at the same time, i.e., during the same placement round.

% RMs should propose server-only alternatives
\paragraph{\gls{inp} alternatives}
The results reported in \autoref{results} show an underutilization of \glspl{resource:physical} depending on the number of tenant requests containing \glspl{resource:composite:inp}.

To maximize the (relative) minimum \gls{resource:physical} utilization, the \gls{rm} would need to adjust the ratio of \gls{inp} requests over \glslink{resource:logical:server}{servers}-only ones.
With the assumption that \glspl{resource:physical:switch} can become the bottleneck of whole the system more rapidly (e.g., less \glslink{resource:physical:switch}{switches} than \glslink{resource:physical:server}{servers} in a fat-tree with $k>5$ or for the scarceness of \glspl{cu} in general),
\glspl{rm} should be able to propose alternatives to \glspl{resource:composite:inp} made out of \glspl{resource:logical:server} or \glspl{resource:composite:server} only, as soon as \glspl{resource:physical:switch} become heavily utilized.
Those alternatives could be stored, for instance, in the same \textit{template database} (\autoref{system_design_overview}) used to translate \glspl{resource:composite:inp} into a set of \glspl{resource:logical:switch}.

\section{Open problems}
Some of the important metrics used in \autoref{evaluation} are still unknown due to the lack of operational \glspl{rm} that handle \gls{inp} requests.

% STC
\glsreset{stc}
In order for \glspl{rm} to be able to propose alternatives, the performance improvement of all \gls{inp} solutions in the \textit{template database} must be reduced to a single number that represents the reduction of \glslink{resource:logical:server}{server} tasks after the introduction of an \gls{resource:composite:inp}: the \gls{stc}.

% Number of needed switch tasks
Moreover, there should be a precise way of determining the number of \glslink{resource:logical:switch}{switch} tasks needed to implement an \gls{resource:composite:inp}, and this could depend on a lot of factors like the set nodes to which the \gls{resource:composite:inp} is connected to in the \gls{etag}, its high-level properties, etc.

% INP service and batch composites
Another key aspect to consider is the type of life cycle that different \gls{inp} solutions might have.
Similarly to \glslink{resource:logical:server}{server} requests, batch (short-term) and service (long-term) jobs may need different scheduling policies.
\gls{inp} solutions should also be categorized on the same basis, e.g., NetChain \cite{netchain} as a service \gls{resource:composite:inp} (since most likely it will be running for a long time, serving multiple \glslink{resource:logical:server}{server} tasks) and Daiet \cite{daiet} or SHArP \cite{sharp} as a batch one.