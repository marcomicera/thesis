Some of the important metrics used in \autoref{evaluation} are still unknown due to the lack of operational \glspl{rm} that handle \gls{inp} requests.

% STC
\glsreset{stc}
In order for \glspl{rm} to be able to propose alternatives, the performance improvement of all \gls{inp} solutions in the \textit{template database} must be reduced to a single number that represents the reduction of \glslink{resource:logical:server}{server} tasks after the introduction of an \gls{resource:composite:inp}: the \gls{stc}.

% Number of needed switch tasks
Moreover, there should be a precise way of determining the number of \glslink{resource:logical:switch}{switch} tasks needed to implement an \gls{resource:composite:inp}, and this could depend on a lot of factors like the set nodes to which the \gls{resource:composite:inp} is connected to in the \gls{etag}, its high-level properties, etc.

% INP service and batch composites
Another key aspect to consider is the type of life cycle that different \gls{inp} solutions might have.
Similarly to \glslink{resource:logical:server}{server} requests, batch (short-term) and service (long-term) jobs may need different scheduling policies.
\gls{inp} solutions should also be categorized on the same basis, e.g., NetChain \cite{netchain} as a service \gls{resource:composite:inp} (since most likely it will be running for a long time, serving multiple \glslink{resource:logical:server}{server} tasks) and Daiet \cite{daiet} or SHArP \cite{sharp} as a batch one.
