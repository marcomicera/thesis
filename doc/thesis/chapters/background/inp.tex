Within this project, in-network processing (INP) refers to the technique that exploits network switches to modify and/or store data packets, without involving any kind of higher-layer devices. Therefore, approaches that make use of middle-boxes do not fall within our definition of INP. This is why INP is different from \textit{active networking} and \textit{Network Function Virtualization} (NFV).