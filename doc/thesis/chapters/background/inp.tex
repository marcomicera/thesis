Within this project, data center \gls{inp} refers to the technique of delegating some parts of the computation to programmable switches, hence reducing servers' workload.
Common operations performed by these devices are data aggregation and key-value storage management.

% Approaches that make use of middle-boxes do not fall within this definition of \gls{inp}.
% This is why \gls{inp} is different from \textit{active networking} and \gls{nfv}.

In contrast to \gls{nfv}, \gls{inp} tries to delegate as much of computation to the network, but both techniques can exploit the advantages of \gls{sdn} for more flexible network management.