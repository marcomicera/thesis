In this thesis, resources are divided by
\begin{mylist}
    \item their level of abstraction and
    \item their type
\end{mylist}.
The glossary reported at the end of this thesis follows a bottom-up approach, starting from those terms belonging to the lowest level of abstraction.

% Glossary made with a simple labeled list
\begin{comment}
    \begin{itemize}
        \item \textbf{Physical resources}:
        \label{gls:resource:physical}
        set of physical hardware component of limited availability within a computer system. 
        
        \begin{itemize}
            \item \textbf{Physical server resources}:
            \label{gls:resource:physical:server}
            resources of physical server machines.
            
            \item \textbf{Physical switch resources}:
            \label{gls:resource:physical:switch}
            resources of physical switches, network accelerators, middle-boxes and of every kind of network device originally intended to forward packets.
        \end{itemize}
        
        \item \textbf{Logical resources}:
        \label{gls:resource:logical}
        logical representation of physical resources.
        
        \begin{itemize}
            \item \textbf{Logical server resources}:
            \label{gls:resource:logical:server}
            virtualized server physical resources, often implemented by means of \glspl{vm}, containers or entire physical servers.
            
            \item \textbf{Logical switch resources}:
            \label{gls:resource:logical:switch}
            logical representation of physical switch resources not mapped to any physical switch device.
            
            \item \textbf{Logical edge resources}:
            \label{gls:resource:logical:edge}
            properties of virtual connections between logical resources, e.g., bandwidth, latency, etc.
        \end{itemize}
        
        \item \textbf{Composites}:
        \label{gls:resource:composite}
        templates describing high-level logical components.
        They can be made out of other composites and logical resources.
        
        \begin{itemize}
            \item \textbf{Server composites}:
            \label{gls:resource:composite:server}
            composites describing high-level server components, e.g., \textit{web server}, \textit{databases}, etc.
            
            \item \textbf{\gls{inp} composites}:
            \label{gls:resource:composite:inp}
            composites describing high-level \gls{inp} applications, e.g., \textit{IncBricks} \cite{incbricks}, \textit{NetChain} \cite{netchain}, etc.
        \end{itemize}
        
        \item \textbf{Resource model}:
        \label{gls:model}
        model capable of describing composites and logical resources.
        The model exposed to tenants and the one internally used by the \gls{rm} could be different.
        
        \begin{itemize}
            \item \textbf{Tenant-side model}:
            \label{gls:model:tenant}
            resource model exposed to tenants by the system \gls{api}.
            
            \item \textbf{\gls{rm}-side model}:
            \label{gls:model:rm}
            resource model internally used by the placement algorithm in order to allocate logical resources.
        \end{itemize}
    \end{itemize}
\end{comment}