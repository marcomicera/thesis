% Site
    % Data center
        % Clusters
            % Cells = Pods
                % Racks
                    % ToR
                    % Servers
                        % Nodes

The elementary computational unit in a data center is called \textit{node}.
As mentioned in \autoref{rm_background}, nodes are being virtualized by means of \glspl{vm} and containers.
Nodes are run on \textit{servers}, which are grouped in \textit{racks}.
Servers within a rack are usually connected through a so-called \gls{tor} switch.
Server racks are then grouped into \textit{cells} (or \textit{pod}), that some times may be special-purpose.
Usually a cell is grouped with few small ones used for testing to form up a \textit{cluster}.
Most of the \glspl{rm} manage resources in one cluster.
One or more clusters form a \textit{data center}, which together form a \textit{site}.