In a data center, resources of any kind are being virtualized in order to achieve higher flexibility, portability, and availability.
Usually, both compute and storage resources are virtualized by means of \glspl{vm} and/or containers.
Flexibility and portability are both automatically achieved thanks to this resource virtualization, resulting in software that can be deployed dynamically, run by multiple platforms and even live migrated; availability is usually simply achieved by not co-locating \glspl{vm} and/or containers within a single power domain.

Nowadays, multiple \glspl{rm} are using different approaches to solve different design issues.
This section examines these existing \glspl{rm} while trying to categorize them based on how they face different scheduling problems.