Firmament \cite{firmament} is a centralized cluster scheduler capable of supporting multi-dimensional resource requirements (\autoref{firmament:multi-dimensions}).
The scheduler finds an embedding solution by solving a min-cost max-flow optimization problem over a graph called \textit{flow network} (\autoref{firmament:flow_network}).

\subsubsection{Flow network} \label{firmament:flow_network}
The flow network is a directed weighted graph that contains both \glslink{resource:physical}{physical} and \glspl{resource:logical}.
Similarly to any other graph-based scheduler (e.g., \cite{ontackling}), a \gls{resource:physical} is connected to a \gls{resource:logical} anytime the former can be allocated on the latter.
In order to reduce the overall amount of arcs, the flow network makes use of \textit{equivalence class aggregates} which group nodes with similar characteristics (e.g., tasks within a job, physical machines with the same micro-architectural topology, etc.).
An aggregate receives arcs from all entities belonging to it.
Physical aggregates can also receive arcs from those logical nodes that can be allocated on any of the physical entity represented by the aggregate.\par
The physical network is faithfully recreated in the graph thanks to aggregates: for instance, different machines belonging to the same rack are connected to the same \textit{rack aggregate}, all racks are connected to the cluster aggregate and so on.
The description does not either stop at the machine level since also sockets and cores are represented in the graph.
Again, inclusion relationships are expressed in terms of arcs.
Special \textit{unscheduled aggregates} (one for each job) receive an arc from every task.\par
The embedding solution is found by running an instance of a min-cost max-flow optimization algorithm over the network flow.
If a task cannot be allocated on any of the physical machines, its flow will be directed to its corresponding unscheduled aggregate.
Arcs' costs are determined by a given scheduling policy (or "cost model"): at the time of writing, Firmament \cite{firmament} supports 9 different scheduling policies.
Intuitively, arcs connecting tasks to unscheduled aggregates will have a higher cost with respect to the ones ending in a physical entity.

\subsubsection{Multi-dimensional resource fitting} \label{firmament:multi-dimensions}
Firmament's \cite{firmament} creator Malte Schwarzkopf introduced the coordinated co-location (CoCo) cost model in his Ph.D. dissertation - Schwarzkopf, M. (2018). \textit{Operating system support for warehouse-scale computing} (Doctoral thesis). \href{https://doi.org/10.17863/CAM.26443}{https://doi.org/10.17863/CAM.26443}.\par
This scheduling policy has an interesting feature: \textit{admission control}.
Basically, a task can be allocated on a physical machine only when it satisfies all task's resource requirements (strict resource fit).
If a task cannot be allocated on any of the physical machines, it cannot be even connected to the cluster aggregate.
Given a \gls{resource:logical} in the network flow, CoCo can efficiently determine all subtrees in the physical topology in which it can be allocated.
This is done by storing in each physical aggregate the minimum and maximum amount of available resources across its children.