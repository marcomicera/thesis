Different \gls{rm} scheduling architectures were introduced in \autoref{rm_architectures}.
This section tries to analyze how could different scheduling architectures manage \gls{inp} resources, highlighting all the drawbacks brought by each one of them.
This brief analysis does not make any assumption regarding the \gls{inp} resource model since its format is out of the scope of this specific section.

\paragraph{Monolithic}
In this architecture, resource requests cannot be concurrent by definition since there is just one logically centralized scheduler.
However, this architecture does not scale due to its intrinsic nature.
Such a scheduling architecture will be used later on to build a greedy \gls{inp}-aware scheduler that will underline the general problems that arise when dealing with \gls{inp} resources (\autoref{inp_aware_rms}).

\paragraph{Two-level}
In the two-level architecture, a centralized node master (or \textit{allocator}) proposes resources to the schedulers.
Let us suppose that the allocator not only proposes \glslink{resource:physical:server}{server resources} but also \gls{inp} ones: all resources must be locked from the instant in which resources are
26offered until the scheduler accepts or denies the offer.
Locking \gls{inp} resources though means locking either \glslink{resource:physical:switch}{network devices' resources} (to perform computation) or links' bandwidth (for guarantees), which can heavily affect the performance of all nodes using these shared network resources.\par
One could think of not locking \gls{inp} resources during the delay introduced by the scheduler when accepting or denying the request, but the allocator would then need to repeat the whole offering process in case another scheduler accepts some \gls{inp} resources that have been acquired by some other scheduler in the meantime.

\paragraph{Shared-state}
A shared-state scheduler could include the \gls{inp} resources state in the \textit{cell state} data structure.
Supposing that tenant applications can request for both \glslink{resource:physical:server}{server} and \gls{inp} resources, schedulers will try to acquire those by modifying the cell state through an atomic commit.
Schedulers that will have to satisfy requests containing supposedly-longer resource request lists are more likely to cause a higher amount of conflicts when attempting to write the share cell state data structure.
This could heavily affect the whole \gls{rm} performance.