\glsreset{vc}
\glsreset{voc}
\glsreset{tivc}

Proposed in the Oktopus \cite{oktopus} paper, a \gls{vc} is a logical one-level tree in which $N$ \glspl{vm} are connected to a single \textit{virtual switch} by a bidirectional link of bandwidth $B$.

\begin{figure}[!htb]
    \centering
    \usebox{\vcfigure}
    \caption{A graphical representation of the \gls{vc} model}
\end{figure}

The virtual switch has a bandwidth of $N \cdot B$, hence the \gls{vc} has no oversubscription, unlike the \gls{voc} model described in \autoref{voc_analysis}.
Authors say that the absence of oversubscription makes it suitable for data-intensive applications.

\subsubsection{Usage}
\paragraph{Kraken \texorpdfstring{\cite{kraken}}{}}
The \gls{vc} is the only resource model used by Kraken \cite{kraken}.
The system allows tenants to update their minimum guarantees in terms of bandwidth and the amount server resources.
However, tenants can only express the number of needed computing resources (called \textit{computing units}) and not their internal requirements (like CPU cores, memory, etc.).
The assumption that all resources are the same is not acceptable in an \gls{inp} scenario.

\paragraph{Oktopus \texorpdfstring{\cite{oktopus}}{}}
Tenant applications can choose one of the three following options to express their request: they can either
\begin{mylist}
    \item stick with the classic resource request by listing individual server resource demands without expressing bandwidth guarantees, accepting to simply get some share of the network resources,
    \item choose a \gls{vc}, most likely used to request resources for data-intensive applications which do not tolerate oversubscribed networks or
    \item specify a \gls{voc} for those applications having more intra-component communication than an inter-component one
\end{mylist}.\par
Yet, when the Oktopus \cite{oktopus} system receives a request expressed with a \gls{vc}, it assumes that \glspl{vm} can be allocated on any server with enough \textit{free \gls{vm} slots}, exactly like Kraken \cite{kraken} with its \textit{computing units}.