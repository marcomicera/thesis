Rather than a fully descriptive resource model, high-level goals specification can be seen as an add-on to the other previously-mentioned models that is worth mentioning for the purpose of this thesis.
Taking Bazaar \cite{bazaar} as an example, one possible high-level goal is the job completion time.
Goals like these eventually need to be translated into classic resource requirements to be correctly interpreted by a \gls{rm}.

\subsubsection{Describing INP solutions}
High-level goals are general and they do not refer to any particular kind of resource.
The use of additional high-level goals for requesting \gls{inp} resources has to be considered during the resource model design phase (\autoref{model_design}) since this concept can be automatically used without the need of modifying its structure.