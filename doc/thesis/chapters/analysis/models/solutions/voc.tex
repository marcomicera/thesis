\glsreset{voc}

Also proposed by Oktopus \cite{oktopus} authors, the \gls{voc} model consists of $N$ \glspl{vm} arranged in groups of size $S$.
Similarly to the \gls{vc} model, \glspl{vm} belonging to the same group are connected to a single virtual switch by a bidirectional link of bandwidth $B$.
Therefore, every virtual switch connecting groups of $S$ \glspl{vm} has a total bandwidth of $N \cdot B$.
All groups are then connected by a unique \textit{root virtual switch}, making this model's topology a two-level logical tree.

\begin{figure}[!htb]
    \centering
    \usebox{\vocfigure}
    \caption{A graphical representation of the \gls{voc} model}
\end{figure}

Links connecting all virtual switches to the root one have an oversubscription factor of $O$, meaning that each of those links has a bandwidth of $S \cdot B / O$.
The \gls{voc} model aims to relax the dense connectivity requirement between all \glspl{vm}, having oversubscription just for inter-group communication.

\subsubsection{Usage}
\paragraph{Oktopus \texorpdfstring{\cite{oktopus}}{}}
Tenants who use a \gls{voc} model for their requests are aware of the bandwidth oversubscription for inter-group communications as they explicitly have to specify the oversubscription factor $O$ in the request.
Oktopus \cite{oktopus} though still does not make any distinction between server resources, making it an unpractical solution for handling \gls{inp} resources.