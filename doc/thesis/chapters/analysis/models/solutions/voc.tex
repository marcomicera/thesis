\glsreset{voc}

\subsubsection{Description}
Also proposed by Oktopus \cite{oktopus} authors, the \gls{voc} model consists in $N$ \glspl{vm} arranged in groups of size $S$.
Similarly to the \gls{vc} model, \glspl{vm} belonging to the same group are connected to a single virtual switch by a bidirectional link of bandwidth $B$.
Therefore, every virtual switch connecting groups of $S$ \glspl{vm} has total bandwidth of $N \cdot B$.
All groups are then connected together by a unique \textit{root virtual switch}, making this model's topology a two-level logical tree.
Links connecting all virtual switches to the root one have an oversubscription factor of $O$, meaning that each of those links has a bandwidth of $S \cdot B / O$.
The \gls{voc} model aims to relax the dense connectivity requirement between all \glspl{vm}, having oversubscription just for inter-group communication.

\subsubsection{Describing INP solutions}
The \gls{voc} model has the advantage of not requiring a virtual switch of bandwidth $N \cdot B$, that could be unpractical when $N$ is particularly large.
When a tenant application issues a resource request using a \gls{voc}, it just requests for the $<N, B, S, O>$ tuple, being respectively
\begin{mylist}
    \item the number of \glspl{vm},
    \item the bandwidth needed for intra-group communication,
    \item the group size and
    \item the oversubscription factor
\end{mylist}.
Still, certain kinds of applications might need different values of $B$ for different groups.
Furthermore, the two-layer logical tree topology might not suit perfectly for applications that require a chain having more than 2 network devices or a tree topology with more than 2 layers.

\subsubsection{Usage}
\paragraph{Oktopus \texorpdfstring{\cite{oktopus}}{}}
Tenants who use a \gls{voc} model for their requests are aware of the bandwidth oversubscription for inter-group communications as they explicitly have to specify the oversubscription factor $O$ in the reuqest.
Oktopus \cite{oktopus} though still does not make any distinction between server resources, making it an unpractical solution for handling \gls{inp} resources.