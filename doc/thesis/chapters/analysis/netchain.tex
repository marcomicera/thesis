NetChain \cite{netchain} is an in-network key-value storage solution.\\
The network device in charge of storing the distributed storage/cache is a programmable switch: this brings an obvious limitation in terms of storage size, which makes NetChain \cite{netchain} an acceptable solution only when a small amount of critical data must be stored in the network data plane (e.g., locks).
NetChain \cite{netchain} also processes queries entirely in the network data plane.

\subsection{Details}
A variant of Chain Replication \cite{chainreplication} is used in the data plane to handle read and write queries and to ensure strong consistency, while reconfiguration operations are handled by the network control plane.\par
The main difference with the standard Chain Replication \cite{chainreplication} protocol is that objects are stored on programmable switches instead of servers.
Switches are logically connected together in order to form an oriented chain: read queries are processed at the end of the chain (the \textit{tail}) while write queries are sent to the \textit{head} of the chain and the state is forwarded along the chain.\par
The key-value store is partitioned among \textit{virtual nodes} using consistent hashing, mapping keys to a hash ring.
Each ring segment is stored by $f + 1$ virtual nodes allocated on different physical switches, hence tolerating faults involving up to $f$ switches.
\textbf{\textit{TO BE CONTINUED}} ...

\subsection{Minimum system requirements}
Network devices must form a chain of length $f$ in order to tolerate $f$ failures. A pair of communicating VMs must be connected to at least one network devices belonging to the chain.\par
Network devices must dedicate some local storage to NetChain \cite{netchain}: more specifically, they need to store a
\begin{mylist}
    \item register array to store values and a
    \item match-action table to store the keys' location in the register array and the corresponding action to be performed
\end{mylist}.