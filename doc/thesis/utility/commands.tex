% !TEX encoding = UTF-8 Unicode
% !TEX TS-program = pdflatex

% Document class
\documentclass[%
    corpo=12pt,
    twoside,
    %    stile=classica,
    oldstyle,
    %    autoretitolo,
    tipotesi=magistrale,
    greek,
    evenboxes,
]{toptesi}
\usepackage[utf8]{inputenc}
\usepackage[T1]{fontenc} % font encoding
\usepackage{lmodern} % font
\pdfminorversion=5

% Table of contents
\usepackage[nottoc,notlof,notlot]{tocbibind} % including references
\setcounter{tocdepth}{2} % depth
\setcounter{secnumdepth}{3} % section numbering depth

% Miscellaneous
% \usepackage{fullpage} % different margins
\usepackage[hidelinks]{hyperref} % links
% \hypersetup{
%     pdfpagemode={UseOutlines},
%     bookmarksopen,
%     pdfstartview={FitH},
%     colorlinks,
%     linkcolor={blue},
%     citecolor={blue},
%     urlcolor={blue}
% }
\usepackage[binary-units=true]{siunitx} % MB, GB, etc.
\usepackage{pdfpages} % import PDFs
\usepackage{verbatim} % multi-line comments

% Checkmarks and x-marks
\usepackage{amssymb}
\usepackage{pifont}
\definecolor{green}{RGB}{0,176,80}
\newcommand{\cmark}{{\color{green}\textbf{\ding{51}}} }
\definecolor{red}{RGB}{222,0,0}
\newcommand{\xmark}{{\color{red}\textbf{\ding{55}}} }

% Adding a dot at the end of paragraph titles
\let\originalparagraph\paragraph
\renewcommand{\paragraph}[2][.]{\originalparagraph{#2#1}}

% Glossaries and acronyms
\usepackage[acronym,toc]{glossaries} % package
% General
\newacronym{rm}{RM}{Resource Manager}
\newacronym{rmf}{RMF}{Resource Management Framework}
\newacronym{vm}{VM}{Virtual Machine}
\newacronym{inp}{INP}{In-Network Processing}
\newacronym{nfv}{NFV}{Network Function Virtualization}
\newacronym{sdn}{SDN}{Software Defined Networking}
\newacronym{tor}{ToR}{Top of Rack}
\newacronym{hpc}{HPC}{High Performance Computing}
\newacronym{dht}{DHT}{Distributed hash table}

% Programming
\newacronym{api}{API}{Application Programming Interface}
\newacronym{mpi}{MPI}{Message Passing Interface}
\newacronym{rpc}{RPC}{Remote Procedure Call}

% Resource models
\newacronym{vc}{VC}{Virtual Cluster}
\newacronym{voc}{VOC}{Virtual Oversubscribed Cluster}
\newacronym{tivc}{TIVC}{Time-Interleaved Virtual Cluster}
\newacronym{tag}{TAG}{Tenant Application Graph}

% SHArP
\newacronym{an}{AN}{Aggregation Node}
\newacronym{am}{AM}{Aggregation Manager}
\newacronym{tca}{TCA}{Target Channel Adapter}
\newacronym{qp}{QP}{Queue Pair}
\newglossaryentry{ibm}{
    name=IBM,
    description=IBM\textsuperscript{\textregistered}
}
\newglossaryentry{switchib2}{
    name=Mellanox's SwitchIB-2,
    description=Mellanox's SwitchIB-2\textsuperscript{TM}
}

% Apache Hadoop YARN
\newglossaryentry{apache}{
    name=Apache,
    description=Apache\textsuperscript{TM}
}
\newglossaryentry{hadoop}{
    name=Apache Hadoop,
    description=\glsdesc{apache} Hadoop\textsuperscript{\textcopyright}
}
\newglossaryentry{yarn_full}{
    name=Apache Hadoop YARN,
    description=\glsdesc{hadoop} YARN \texorpdfstring{\cite{yarn}}{}
}
\newglossaryentry{yarn}{
    name=Apache YARN,
    description=\glsdesc{apache} YARN \texorpdfstring{\cite{yarn}}{}
} % acronyms
% Glossary definition
\newglossary*{resources}{Resources glossary}

\newglossaryentry{resource:physical}{
    type=resources,
    name=Physical resource,
    text=physical resource,
    description={physical hardware component of limited availability within a physical machine}
}

    \newglossaryentry{resource:physical:server}{
        type=resources,
        name=Physical server resource,
        text=physical server resource,
        parent=resource:physical,
        description={resource of a physical server machine}
    }
    
    \newglossaryentry{resource:physical:switch}{
        type=resources,
        name=Physical switch resources,
        text=physical switch resources,
        parent=resource:physical,
        description={resource of a physical switche, network accelerator, middle-box and of every kind of network device originally intended to forward packets}
    }
    
\newglossaryentry{resource:logical}{
    type=resources,
    name=Logical resource,
    text=logical resource,
    description={logical representation of a physical resource}
}

    \newglossaryentry{resource:logical:server}{
        type=resources,
        name=Logical server resource,
        text=logical server resource,
        parent=resource:logical,
        description={virtualized server physical resource, often implemented by means of a \gls{vm}, container or an entire physical server}
    }
    
    \newglossaryentry{resource:logical:switch}{
        type=resources,
        name=Logical switch resource,
        text=logical switch resource,
        parent=resource:logical,
        description={logical representation of a physical switch resource not mapped to any physical switch device}
    }
    
    \newglossaryentry{resource:logical:edge}{
        type=resources,
        name=Logical edge resource,
        text=logical edge resource,
        parent=resource:logical,
        description={properties of virtual connections between two logical resources, e.g., bandwidth, latency, etc}
    }
    
\newglossaryentry{resource:composite}{
    type=resources,
    name=Composite,
    text=composite,
    description={template describing a high-level logical component.
        It can be made out of other composites and/or logical resources.}
}

    \newglossaryentry{resource:composite:server}{
        type=resources,
        name=Server composite,
        text=server composite,
        parent=resource:composite,
        description={composite describing a high-level server component, e.g., \textit{web server}, \textit{databases}, etc}
    }
    
    \newglossaryentry{resource:composite:inp}{
        type=resources,
        name=\gls{inp} composite,
        text=\texorpdfstring{\gls{inp}}{INP} composite,
        parent=resource:composite,
        description={composite describing a high-level \gls{inp} application, e.g., \textit{IncBricks} \cite{incbricks}, \textit{NetChain} \cite{netchain}, etc}
    }
    
\newglossaryentry{model}{
    type=resources,
    name=Resource model,
    text=resource model,
    description={model capable of describing composites and logical resources.
        The model exposed to tenants and the one internally used by the \gls{rm} could be different}
}

    \newglossaryentry{model:tenant}{
        type=resources,
        name=Tenant-side model,
        text=tenant-side model,
        parent=model,
        description={resource model exposed to tenants by the system \gls{api}}
    }
    
    \newglossaryentry{model:rm}{
        type=resources,
        name=\gls{rm}-side model,
        text=\texorpdfstring{\gls{rm}}{RM}-side model,
        parent=model,
        description={resource model internally used by the placement algorithm in order to allocate logical resources}
    } % resources glossary
\makenoidxglossaries % computing the glossary and acronyms list

% In-line lists
\usepackage[inline]{enumitem}
\newlist{mylist}{enumerate*}{1}
\setlist[mylist]{label=(\roman*)}

% References sections
\renewcommand{\sectionautorefname}{\S}
\renewcommand{\subsectionautorefname}{\S}
\renewcommand{\subsubsectionautorefname}{\S}
\renewcommand\bibname{References}

% Capitalized references
\usepackage{cleveref}

% Footnotes with symbols
\usepackage[symbol]{footmisc}
\renewcommand{\thefootnote}{\fnsymbol{footnote}}

% Blank pages
\usepackage{afterpage}
\newcommand\blankpage{%
    \null
    \thispagestyle{empty}%
    \addtocounter{page}{-1}%
    \newpage
}

% Comments
\newcommand\pnote[1]{\textit{\textcolor{blue}{[p@]: #1}}}
\newcommand\ma[1]{\textit{\textcolor{purple}{[marcel]: #1}}}
\newcommand\lin[1]{\textit{\textcolor{green!55!blue}{[lin]: #1}}}
\newcommand\marco[1]{\textit{\textcolor{red}{[marco]: #1}}}
\newcommand\fulvio[1]{\textit{\textcolor{orange}{[fulvio]: #1}}}