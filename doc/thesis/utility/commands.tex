% !TEX encoding = UTF-8 Unicode
% !TEX TS-program = pdflatex

% Document class
\documentclass[%
    corpo=12pt,
    twoside,
    %    stile=classica,
    oldstyle,
    %    autoretitolo,
    tipotesi=magistrale,
    greek,
    evenboxes,
]{toptesi}
\usepackage[utf8]{inputenc}
\usepackage[T1]{fontenc} % font encoding
\usepackage{lmodern} % font
\pdfminorversion=5

% Table of contents
\usepackage[nottoc,notlof,notlot]{tocbibind} % including references
\setcounter{tocdepth}{2} % depth
\setcounter{secnumdepth}{3} % section numbering depth

% Front page
\renewcommand\IDlabel{\\\quad Student ID:\xspace}

% Miscellaneous
% \usepackage{fullpage} % different margins
\usepackage[hidelinks]{hyperref} % links
% \hypersetup{
%     pdfpagemode={UseOutlines},
%     bookmarksopen,
%     pdfstartview={FitH},
%     colorlinks,
%     linkcolor={blue},
%     citecolor={blue},
%     urlcolor={blue}
% }
\usepackage[binary-units=true]{siunitx} % MB, GB, etc.
\usepackage{pdfpages} % import PDFs
\usepackage{verbatim} % multi-line comments

% Checkmarks and x-marks
\usepackage{amssymb}
\usepackage{pifont}
\definecolor{green}{RGB}{0,176,80}
\newcommand{\cmark}{{\color{green}\textbf{\ding{51}}} }
\definecolor{red}{RGB}{222,0,0}
\newcommand{\xmark}{{\color{red}\textbf{\ding{55}}} }

% Adding a dot at the end of paragraph titles
\let\originalparagraph\paragraph
\renewcommand{\paragraph}[2][.]{\originalparagraph{#2#1}}

% Glossaries and acronyms
\usepackage[acronym,toc]{glossaries} % package
% General
\newacronym{api}{API}{Application Programming Interface}
\newacronym{rm}{RM}{Resource Manager}
\newacronym{vm}{VM}{Virtual Machine}
\newacronym{inp}{INP}{In-Network Processing}
\newacronym{nfv}{NFV}{Network Function Virtualization}
\newacronym{sdn}{SDN}{Software Defined Networking}
\newacronym{voc}{VOC}{Virtual Oversubscribed Cluster}
\newacronym{tor}{ToR}{Top of Rack}
\newacronym{mpi}{MPI}{Message Passing Interface}
\newacronym{hpc}{HPC}{High Performance Computing}

% CloudMirror
\newacronym{tag}{TAG}{Tenant Application Graph}

% SHArP
\newacronym{an}{AN}{Aggregation Node}
\newacronym{am}{AM}{Aggregation Manager}
\newacronym{tca}{TCA}{Target Channel Adapter}
\newacronym{qp}{QP}{Queue Pair} % acronyms
\paragraph{Data center architecture} \label{dc_architecture}
% Data center fat-tree
    % Also network resources
        % Uniform compute units (CU) and properties
This simulator emulates \glspl{resource:physical:switch} besides \glslink{resource:physical:server}{server ones}.
It can support any kind of data center network architecture.
Specifically, a \textit{fat-tree} has been used for this evaluation, which is a very common topology for data centers to use. An example of a fat-tree topology is depicted in \autoref{fig:fattree}.
\begin{figure}[!htb]
    \centering
    \usebox{\fattree}
    \caption{A fat-tree topology with 4 \textit{pods}}
    \label{fig:fattree}
\end{figure}

This fat-tree has three layers of \glslink{resource:physical:switch}{switches}: \textit{core}, \textit{aggregation} and the last one which is usually called \textit{edge}, \textit{layer}, \textit{access} or just simply \gls{tor} \glslink{resource:physical:switch}{switches}.
Being $k$ the number of \textit{pods} in the topology, a fat-tree contains $(k/2)^2$ core \glslink{resource:physical:switch}{switches}, $k^2/2$ aggregation \glslink{resource:physical:switch}{switches}, $k^2/2$ \gls{tor} \glslink{resource:physical:switch}{switches}, and supports up to $k^3/4$ \glslink{resource:physical:server}{servers}.
Being then $5k^2/4$ the total amount of \glslink{resource:physical:switch}{switches} in a fat-tree, \glslink{resource:physical:server}{servers} become more abundant when $k>5$.

\paragraph{\Glspl{resource:physical:switch}} \label{simulator_switch_resources}
\glsreset{cu}
For the sake of simplicity, \glslink{resource:physical:switch}{physical switches} have a single numerical dimension called \gls{cu}.
This assumption does not affect this evaluation's results and it makes the scheduling algorithm a bit simpler.
Increasing the number of \glslink{resource:physical:switch}{switch} dimensions is trivial since the simulator already supports multiple dimensions for \glslink{resource:physical:server}{servers}, namely CPU and memory.
\Glspl{resource:logical:edge} have been ignored to simplify the placement algorithm.

\glslink{resource:physical:switch}{Switches} are also characterized by a \textit{property map}, that ultimately suggests which kinds of \gls{inp} solutions they are able to run.
In this evaluation, properties and \gls{inp} solutions coincide (e.g., \glslink{resource:physical:switch}{switch} $A$ supports \gls{inp} solutions $X$, $Y$, and $Z$), but these properties might be more appropriately extended to any other kind of hardware property that distinguish \glslink{resource:physical:switch}{switches} in the data center (e.g., CPU architecture, supported data plane programming language, etc.).
In general, a switch task requesting for property $P$ will only be allocated on a \glslink{resource:physical:switch}{switch} if it supports that same property $P$. % resources glossary
\makenoidxglossaries % computing the glossary and acronyms list

% In-line lists
\usepackage[inline]{enumitem}
\newlist{mylist}{enumerate*}{1}
\setlist[mylist]{label=(\roman*)}

% References sections
\renewcommand{\sectionautorefname}{\S}
\renewcommand{\subsectionautorefname}{\S}
\renewcommand{\subsubsectionautorefname}{\S}
\renewcommand\bibname{References}

% Capitalized references
\usepackage{cleveref}

% Footnotes with symbols
\usepackage[symbol]{footmisc}
\renewcommand{\thefootnote}{\fnsymbol{footnote}}

% Blank pages
\usepackage{afterpage}
\newcommand\blankpage{%
    \null
    \thispagestyle{empty}%
    \addtocounter{page}{-1}%
    \newpage
}

% Comments
\newcommand\pnote[1]{\textit{\textcolor{blue}{[p@]: #1}}}
\newcommand\ma[1]{\textit{\textcolor{purple}{[marcel]: #1}}}
\newcommand\lin[1]{\textit{\textcolor{green!55!blue}{[lin]: #1}}}
\newcommand\marco[1]{\textit{\textcolor{red}{[marco]: #1}}}
\newcommand\fulvio[1]{\textit{\textcolor{orange}{[fulvio]: #1}}}