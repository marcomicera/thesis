% Commands
% !TEX encoding = UTF-8 Unicode
% !TEX TS-program = pdflatex

% Document class
\documentclass[%
    corpo=12pt,
    twoside,
    %    stile=classica,
    oldstyle,
    %    autoretitolo,
    tipotesi=magistrale,
    greek,
    evenboxes,
]{toptesi}
\usepackage[utf8]{inputenc}
\usepackage[T1]{fontenc} % font encoding
\usepackage{lmodern} % font
\pdfminorversion=5

% Table of contents
\usepackage[nottoc,notlof,notlot]{tocbibind} % including references
\setcounter{tocdepth}{2} % depth
\setcounter{secnumdepth}{3} % section numbering depth

% Front page
\renewcommand\IDlabel{\\\quad Student ID:\xspace}

% Miscellaneous
% \usepackage{fullpage} % different margins
\usepackage[hidelinks]{hyperref} % links
% \hypersetup{
%     pdfpagemode={UseOutlines},
%     bookmarksopen,
%     pdfstartview={FitH},
%     colorlinks,
%     linkcolor={blue},
%     citecolor={blue},
%     urlcolor={blue}
% }
\usepackage[binary-units=true]{siunitx} % MB, GB, etc.
\usepackage{pdfpages} % import PDFs
\usepackage{verbatim} % multi-line comments

% Checkmarks and x-marks
\usepackage{amssymb}
\usepackage{pifont}
\definecolor{green}{RGB}{0,176,80}
\newcommand{\cmark}{{\color{green}\textbf{\ding{51}}} }
\definecolor{red}{RGB}{222,0,0}
\newcommand{\xmark}{{\color{red}\textbf{\ding{55}}} }

% Adding a dot at the end of paragraph titles
\let\originalparagraph\paragraph
\renewcommand{\paragraph}[2][.]{\originalparagraph{#2#1}}

% Glossaries and acronyms
\usepackage[acronym,toc]{glossaries} % package
% General
\newacronym{api}{API}{Application Programming Interface}
\newacronym{rm}{RM}{Resource Manager}
\newacronym{vm}{VM}{Virtual Machine}
\newacronym{inp}{INP}{In-Network Processing}
\newacronym{nfv}{NFV}{Network Function Virtualization}
\newacronym{sdn}{SDN}{Software Defined Networking}
\newacronym{voc}{VOC}{Virtual Oversubscribed Cluster}
\newacronym{tor}{ToR}{Top of Rack}
\newacronym{mpi}{MPI}{Message Passing Interface}
\newacronym{hpc}{HPC}{High Performance Computing}

% CloudMirror
\newacronym{tag}{TAG}{Tenant Application Graph}

% SHArP
\newacronym{an}{AN}{Aggregation Node}
\newacronym{am}{AM}{Aggregation Manager}
\newacronym{tca}{TCA}{Target Channel Adapter}
\newacronym{qp}{QP}{Queue Pair} % acronyms
\paragraph{Data center architecture} \label{dc_architecture}
% Data center fat-tree
    % Also network resources
        % Uniform compute units (CU) and properties
This simulator emulates \glspl{resource:physical:switch} besides \glslink{resource:physical:server}{server ones}.
It can support any kind of data center network architecture.
Specifically, a \textit{fat-tree} has been used for this evaluation, which is a very common topology for data centers to use. An example of a fat-tree topology is depicted in \autoref{fig:fattree}.
\begin{figure}[!htb]
    \centering
    \usebox{\fattree}
    \caption{A fat-tree topology with 4 \textit{pods}}
    \label{fig:fattree}
\end{figure}

This fat-tree has three layers of \glslink{resource:physical:switch}{switches}: \textit{core}, \textit{aggregation} and the last one which is usually called \textit{edge}, \textit{layer}, \textit{access} or just simply \gls{tor} \glslink{resource:physical:switch}{switches}.
Being $k$ the number of \textit{pods} in the topology, a fat-tree contains $(k/2)^2$ core \glslink{resource:physical:switch}{switches}, $k^2/2$ aggregation \glslink{resource:physical:switch}{switches}, $k^2/2$ \gls{tor} \glslink{resource:physical:switch}{switches}, and supports up to $k^3/4$ \glslink{resource:physical:server}{servers}.
Being then $5k^2/4$ the total amount of \glslink{resource:physical:switch}{switches} in a fat-tree, \glslink{resource:physical:server}{servers} become more abundant when $k>5$.

\paragraph{\Glspl{resource:physical:switch}} \label{simulator_switch_resources}
\glsreset{cu}
For the sake of simplicity, \glslink{resource:physical:switch}{physical switches} have a single numerical dimension called \gls{cu}.
This assumption does not affect this evaluation's results and it makes the scheduling algorithm a bit simpler.
Increasing the number of \glslink{resource:physical:switch}{switch} dimensions is trivial since the simulator already supports multiple dimensions for \glslink{resource:physical:server}{servers}, namely CPU and memory.
\Glspl{resource:logical:edge} have been ignored to simplify the placement algorithm.

\glslink{resource:physical:switch}{Switches} are also characterized by a \textit{property map}, that ultimately suggests which kinds of \gls{inp} solutions they are able to run.
In this evaluation, properties and \gls{inp} solutions coincide (e.g., \glslink{resource:physical:switch}{switch} $A$ supports \gls{inp} solutions $X$, $Y$, and $Z$), but these properties might be more appropriately extended to any other kind of hardware property that distinguish \glslink{resource:physical:switch}{switches} in the data center (e.g., CPU architecture, supported data plane programming language, etc.).
In general, a switch task requesting for property $P$ will only be allocated on a \glslink{resource:physical:switch}{switch} if it supports that same property $P$. % resources glossary
\makenoidxglossaries % computing the glossary and acronyms list

% In-line lists
\usepackage[inline]{enumitem}
\newlist{mylist}{enumerate*}{1}
\setlist[mylist]{label=(\roman*)}

% References sections
\renewcommand{\sectionautorefname}{\S}
\renewcommand{\subsectionautorefname}{\S}
\renewcommand{\subsubsectionautorefname}{\S}
\renewcommand\bibname{References}

% Capitalized references
\usepackage{cleveref}

% Footnotes with symbols
\usepackage[symbol]{footmisc}
\renewcommand{\thefootnote}{\fnsymbol{footnote}}

% Blank pages
\usepackage{afterpage}
\newcommand\blankpage{%
    \null
    \thispagestyle{empty}%
    \addtocounter{page}{-1}%
    \newpage
}

% Comments
\newcommand\pnote[1]{\textit{\textcolor{blue}{[p@]: #1}}}
\newcommand\ma[1]{\textit{\textcolor{purple}{[marcel]: #1}}}
\newcommand\lin[1]{\textit{\textcolor{green!55!blue}{[lin]: #1}}}
\newcommand\marco[1]{\textit{\textcolor{red}{[marco]: #1}}}
\newcommand\fulvio[1]{\textit{\textcolor{orange}{[fulvio]: #1}}}

% Document start
\begin{document}
\noindent

% Title
% \begin{titlepage}
%     \begin{center}
%         \vspace*{7cm}
 
%         \Huge
%         \textbf{Data center resource management for in-network processing}
 
%         \vspace{1cm}
        
%         \huge
%         Marco Micera
 
%         \vfill
 
%         \LARGE
%         \textit{Politecnico di Torino, TU Darmstadt}
%     \end{center}
% \end{titlepage}

\begin{ThesisTitlePage}
    % Per cambiare la dicitura sopra la lista dei laureandi decommentare
    % la riga seguente, cambiando le 4 parole in modo consistente
    %
    \TitoloListaCandidati{Studente,Studenti,Studentessa,Studentesse}
    %
    \ateneo{Politecnico di Torino, TU Darmstadt}
    %
    % Non tutte le università hanno un nome proprio
    % \nomeateneo{DAUIN - Department of Control and Computer Engineering, DSP - Distributed Systems Programming Group}
    %
    \struttura[III]{Matematica, Fisica e~Scienze Naturali}
    %\Materia{Remote sensing}
    \titolo{Data center resource management for in-network processing}% per la laurea quinquennale e il dottorato
    % \sottotitolo{Metodo dei satelliti medicei}% per la laurea quinquennale e il dottorato
    %
    %%%%%%% Corso degli studi
    \corsodilaurea{Computer Engineering}% per la laurea
    
    %%%%%%% L'eventuale numero di matricola va fra parentesi quadre
    %\show\Candidato
    \def\Candidato{Candidate}
    %\show\Candidato
    \candidato{Marco \textsc{Micera}}[253157] 
    %\secondocandidato{Evangelista \textsc{Torricelli}}[123457]
    
    %%%%%%% Relatori o supervisori
    %
    \def\Relatori{Supervisors}
        \relatore{Prof. Fulvio Risso}
        \secondorelatore{Prof. Patrick Eugster}
        \terzorelatore{M.Sc. Marcel Bl{\"o}cher}
    % 
    %%%%%%% Per mettere altri relatori consultare toptesi-it.pdf
    
    %%%%%%% Tutore
    % \tutoreaziendale{dott.\ ing.\ Giovanni Giacosa}
    % \NomeTutoreAziendale{Supervisore aziendale\\Centro Ricerche FIAT}
    
    %%%%%%% Seduta dell'esame
    %\sedutadilaurea{Agosto 1615}
    %%%%%%%% oppure:
    \sedutadilaurea{April, 2020}% 
    
    %%%%%%% Logo della sede
    % \logosede{logodue}% 
    \end{ThesisTitlePage}


%
% Compliant to the summary needed by the ICM/ETF board secretary
%
% 1. title, graduand name, supervisors names
% 2. thesis purpose and brief description of the research area
% 3. personal contribution and results obtained
%

% Sections
\section{Info}
\textbf{Title}: Data center resource management for in-network processing\\
\textbf{Graduand}: Marco Micera\\
\textbf{Supervisors}: Prof. Fulvio Risso  \footnote[2]{\label{polito} Computer Networks Group, Politecnico di Torino, Italy}, Prof. Patrick Eugster \footnote[3]{\label{tuda} Distributed Systems Programming Group, Technische Universit{\"a}t Darmstadt, Germany}, M.Sc. Marcel Bl{\"o}cher \footref{tuda}\\
\textbf{Research areas}: cloud computing, distributed systems, in-network processing

\section{Thesis purpose}
Nowadays there exist several \gls{inp} solutions that allow tenants to improve their application performance in terms of different metrics: \textsc{Daiet} \cite{daiet} inventors claim to achieve an 86.9\%-89.3\% traffic reduction by performing data aggregation entirely in the network data plane.
Other solutions like {NetChain \cite{netchain} and IncBricks \cite{incbricks} let programmable switches store data and process queries in order to cut end-to-end latency.
It is now even possible to provide guarantees to applications with specific requirements: for instance, CloudMirror \cite{cloudmirror} enables applications to reserve a minimum bandwidth.\par
For the time being, it seems that there is still no valid resource allocation algorithm that takes into account the presence of a network having a data plane that supports (partially or completely) \gls{inp}.
This thesis has mainly two goals:
\begin{mylist}
    \item model and evaluate an \gls{api} through which applications can ask for \gls{inp} resources and % while providing guarantees (e.g., bandwidth). 
    \item argue the importance of a scheduler which is able to reject \gls{inp} requests and propose their server-only equivalent when needed (e.g., high switch utilization) 
\end{mylist}.

\subsection{Modeling \texorpdfstring{\glsentryshort{inp}}{INP} resources}
One of the two goals of this Master's thesis consists in investigating how to model \gls{inp} resources and how to integrate them into RMs.
In order to offer \gls{inp} services to a tenant application, the latter should be able to ask for \gls{inp} resources through an \gls{api}.
To do that, \gls{inp} resources must be modeled not only to support currently existing \gls{inp} solutions such as \cite{daiet} \cite{netchain} \cite{incbricks} \cite{sharp}, but also future ones. 
It may also be convenient to derive a single model to describe both server and \gls{inp} resources.

Classic tenant application requests can often be modeled as a key-value data structure.
CloudMirror \cite{cloudmirror} requires a \gls{tag} as an input, which is a directed graph where each vertex represents an application component and links' weights represent the minimum requested bandwidth.
One possible model could be based on a \gls{tag}, describing network resources and \gls{inp} services as vertexes or links.
Tenant applications could either use the same model used within the data center or a simplified one, adding another level of abstraction.


\subsection{\texorpdfstring{\glsentryshort{inp}}{INP}-aware \texorpdfstring{\glsentrylongpl{rm}}{Resource Managers}} \label{inp_aware_rms}
\glsreset{rm}
In order for everything to work, a network-aware placement algorithm in the \glsentrylong{rm} should be able to consider \gls{inp} and \glslink{resource:logical:server}{server} resources conjunctly: this brings new challenges in the field of resource management as there are currently no \glspl{rm} doing this.
One problem that could arise is due to the fact that \gls{inp} resources are typically very limited in a data center: this thesis argues the importance of an \gls{rm} which is flexible enough to propose alternatives based on the current utilization of \gls{inp} and \glslink{resource:logical:server}{server} resources, since one kind of \glslink{resource:physical}{physical resource} type can become the bottleneck for the other.

\section{Personal contribution and obtained results}

\subsection{System design}

\subsubsection{\texorpdfstring{\Glsentrytext{resource:composite}}{Composite}s translation methods}

\subsection{Resource model}

\subsection{Generic groups}

\subsection{Simulation}

\subsubsection{Results}

\subsection{Conclusions}

% References
\bibliographystyle{abbrv}
\bibliography{../thesis/utility/references}

% Document end
\end{document}