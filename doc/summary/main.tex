% Commands
% Document class
\documentclass[a4paper, 11pt]{report}
\usepackage[utf8]{inputenc}

% Table of contents
\usepackage[nottoc,notlof,notlot]{tocbibind} % including references
\setcounter{tocdepth}{2} % depth
\setcounter{secnumdepth}{3} % section numbering depth

% Miscellaneous
\usepackage{fullpage} % different margins
\usepackage{hyperref} % links
\usepackage[binary-units=true]{siunitx} % MB, GB, etc.
\usepackage{pdfpages} % import PDFs
\usepackage{verbatim} % multi-line comments

% Checkmarks and xmarks
\usepackage{amssymb}
\usepackage{pifont}
\definecolor{green}{RGB}{0,176,80}
\newcommand{\cmark}{{\color{green}\textbf{\ding{51}}} }
\definecolor{red}{RGB}{222,0,0}
\newcommand{\xmark}{{\color{red}\textbf{\ding{55}}} }

% Adding a dot at the end of paragraph titles
\let\originalparagraph\paragraph
\renewcommand{\paragraph}[2][.]{\originalparagraph{#2#1}}

% Glossaries and acronyms
% section,numberedsection=autolabel
\usepackage[acronym,toc]{glossaries} % package
% General
\newacronym{rm}{RM}{Resource Manager}
\newacronym{rmf}{RMF}{Resource Management Framework}
\newacronym{vm}{VM}{Virtual Machine}
\newacronym{inp}{INP}{In-Network Processing}
\newacronym{nfv}{NFV}{Network Function Virtualization}
\newacronym{sdn}{SDN}{Software Defined Networking}
\newacronym{tor}{ToR}{Top of Rack}
\newacronym{hpc}{HPC}{High Performance Computing}
\newacronym{dht}{DHT}{Distributed hash table}

% Programming
\newacronym{api}{API}{Application Programming Interface}
\newacronym{mpi}{MPI}{Message Passing Interface}
\newacronym{rpc}{RPC}{Remote Procedure Call}

% Resource models
\newacronym{vc}{VC}{Virtual Cluster}
\newacronym{voc}{VOC}{Virtual Oversubscribed Cluster}
\newacronym{tivc}{TIVC}{Time-Interleaved Virtual Cluster}
\newacronym{tag}{TAG}{Tenant Application Graph}

% SHArP
\newacronym{an}{AN}{Aggregation Node}
\newacronym{am}{AM}{Aggregation Manager}
\newacronym{tca}{TCA}{Target Channel Adapter}
\newacronym{qp}{QP}{Queue Pair}
\newglossaryentry{ibm}{
    name=IBM,
    description=IBM\textsuperscript{\textregistered}
}
\newglossaryentry{switchib2}{
    name=Mellanox's SwitchIB-2,
    description=Mellanox's SwitchIB-2\textsuperscript{TM}
}

% Apache Hadoop YARN
\newglossaryentry{apache}{
    name=Apache,
    description=Apache\textsuperscript{TM}
}
\newglossaryentry{hadoop}{
    name=Apache Hadoop,
    description=\glsdesc{apache} Hadoop\textsuperscript{\textcopyright}
}
\newglossaryentry{yarn_full}{
    name=Apache Hadoop YARN,
    description=\glsdesc{hadoop} YARN \texorpdfstring{\cite{yarn}}{}
}
\newglossaryentry{yarn}{
    name=Apache YARN,
    description=\glsdesc{apache} YARN \texorpdfstring{\cite{yarn}}{}
} % acronyms
% Glossary definition
\newglossary*{resources}{Resources glossary}

\newglossaryentry{resource:physical}{
    type=resources,
    name=Physical resource,
    text=physical resource,
    description={physical hardware component of limited availability within a physical machine}
}

    \newglossaryentry{resource:physical:server}{
        type=resources,
        name=Physical server resource,
        text=physical server resource,
        parent=resource:physical,
        description={resource of a physical server machine}
    }
    
    \newglossaryentry{resource:physical:switch}{
        type=resources,
        name=Physical switch resources,
        text=physical switch resources,
        parent=resource:physical,
        description={resource of a physical switche, network accelerator, middle-box and of every kind of network device originally intended to forward packets}
    }
    
\newglossaryentry{resource:logical}{
    type=resources,
    name=Logical resource,
    text=logical resource,
    description={logical representation of a physical resource}
}

    \newglossaryentry{resource:logical:server}{
        type=resources,
        name=Logical server resource,
        text=logical server resource,
        parent=resource:logical,
        description={virtualized server physical resource, often implemented by means of a \gls{vm}, container or an entire physical server}
    }
    
    \newglossaryentry{resource:logical:switch}{
        type=resources,
        name=Logical switch resource,
        text=logical switch resource,
        parent=resource:logical,
        description={logical representation of a physical switch resource not mapped to any physical switch device}
    }
    
    \newglossaryentry{resource:logical:edge}{
        type=resources,
        name=Logical edge resource,
        text=logical edge resource,
        parent=resource:logical,
        description={properties of virtual connections between two logical resources, e.g., bandwidth, latency, etc}
    }
    
\newglossaryentry{resource:composite}{
    type=resources,
    name=Composite,
    text=composite,
    description={template describing a high-level logical component.
        It can be made out of other composites and/or logical resources.}
}

    \newglossaryentry{resource:composite:server}{
        type=resources,
        name=Server composite,
        text=server composite,
        parent=resource:composite,
        description={composite describing a high-level server component, e.g., \textit{web server}, \textit{databases}, etc}
    }
    
    \newglossaryentry{resource:composite:inp}{
        type=resources,
        name=\gls{inp} composite,
        text=\texorpdfstring{\gls{inp}}{INP} composite,
        parent=resource:composite,
        description={composite describing a high-level \gls{inp} application, e.g., \textit{IncBricks} \cite{incbricks}, \textit{NetChain} \cite{netchain}, etc}
    }
    
\newglossaryentry{model}{
    type=resources,
    name=Resource model,
    text=resource model,
    description={model capable of describing composites and logical resources.
        The model exposed to tenants and the one internally used by the \gls{rm} could be different}
}

    \newglossaryentry{model:tenant}{
        type=resources,
        name=Tenant-side model,
        text=tenant-side model,
        parent=model,
        description={resource model exposed to tenants by the system \gls{api}}
    }
    
    \newglossaryentry{model:rm}{
        type=resources,
        name=\gls{rm}-side model,
        text=\texorpdfstring{\gls{rm}}{RM}-side model,
        parent=model,
        description={resource model internally used by the placement algorithm in order to allocate logical resources}
    } % resources glossary
\makeglossaries % computing the glossary and acronyms list

% In-line lists
\usepackage[inline]{enumitem}
\newlist{mylist}{enumerate*}{1}
\setlist[mylist]{label=(\roman*)}

% References sections
\renewcommand{\sectionautorefname}{\S}
\renewcommand{\subsectionautorefname}{\S}
\renewcommand{\subsubsectionautorefname}{\S}
\renewcommand\bibname{References}

% Capitalized references
\usepackage{cleveref}

% Footnotes with symbols
\usepackage[symbol]{footmisc}
\renewcommand{\thefootnote}{\fnsymbol{footnote}}

% Document start
\begin{document}
\noindent

% Title
% \begin{titlepage}
%     \begin{center}
%         \vspace*{7cm}
 
%         \Huge
%         \textbf{Data center resource management for in-network processing}
 
%         \vspace{1cm}
        
%         \huge
%         Marco Micera
 
%         \vfill
 
%         \LARGE
%         \textit{Politecnico di Torino, TU Darmstadt}
%     \end{center}
% \end{titlepage}

\begin{ThesisTitlePage}
    % Per cambiare la dicitura sopra la lista dei laureandi decommentare
    % la riga seguente, cambiando le 4 parole in modo consistente
    %
    \TitoloListaCandidati{Studente,Studenti,Studentessa,Studentesse}
    %
    \ateneo{Politecnico di Torino, TU Darmstadt}
    %
    % Non tutte le università hanno un nome proprio
    % \nomeateneo{DAUIN - Department of Control and Computer Engineering, DSP - Distributed Systems Programming Group}
    %
    \struttura[III]{Matematica, Fisica e~Scienze Naturali}
    %\Materia{Remote sensing}
    \titolo{Data center resource management for in-network processing}% per la laurea quinquennale e il dottorato
    % \sottotitolo{Metodo dei satelliti medicei}% per la laurea quinquennale e il dottorato
    %
    %%%%%%% Corso degli studi
    \corsodilaurea{Computer Engineering}% per la laurea
    
    %%%%%%% L'eventuale numero di matricola va fra parentesi quadre
    %\show\Candidato
    \def\Candidato{Candidate}
    %\show\Candidato
    \candidato{Marco \textsc{Micera}}[253157] 
    %\secondocandidato{Evangelista \textsc{Torricelli}}[123457]
    
    %%%%%%% Relatori o supervisori
    %
    \def\Relatori{Supervisors}
        \relatore{Prof. Fulvio Risso}
        \secondorelatore{Prof. Patrick Eugster}
        \terzorelatore{M.Sc. Marcel Bl{\"o}cher}
    % 
    %%%%%%% Per mettere altri relatori consultare toptesi-it.pdf
    
    %%%%%%% Tutore
    % \tutoreaziendale{dott.\ ing.\ Giovanni Giacosa}
    % \NomeTutoreAziendale{Supervisore aziendale\\Centro Ricerche FIAT}
    
    %%%%%%% Seduta dell'esame
    %\sedutadilaurea{Agosto 1615}
    %%%%%%%% oppure:
    \sedutadilaurea{\textsc{Anno~accademico} 2019-2020}% 
    
    %%%%%%% Logo della sede
    % \logosede{logodue}% 
    \end{ThesisTitlePage}


%
% Compliant to the summary needed by the ICM/ETF board secretary
%
% 1. title, graduand name, supervisors names
% 2. thesis purpose and brief description of the research area
% 3. personal contribution and results obtained
%

% Sections
\section{Info}
\textbf{Title}: Data center resource management for in-network processing\\
\textbf{Graduand}: Marco Micera\\
\textbf{Supervisors}: Prof. Fulvio Risso  \footnote[2]{\label{polito} Computer Networks Group, Politecnico di Torino, Italy}, Prof. Patrick Eugster \footnote[3]{\label{tuda} Distributed Systems Programming Group, Technische Universit{\"a}t Darmstadt, Germany}, M.Sc. Marcel Bl{\"o}cher \footref{tuda}\\
\textbf{Research areas}: cloud computing, distributed systems, in-network processing

\section{Thesis purpose}
Data centers distributed systems can nowadays make use of in-network computation to improve several factors: \textsc{Daiet} \cite{daiet} inventors claim to achieve a 86.9\%-89.3\% traffic reduction by performing data aggregation entirely in the network data plane.
Other solutions like \textsc{NetChain} \cite{netchain} and \textsc{IncBricks} \cite{incbricks} let programmable switches store data and process queries in order to cut end-to-end latency.
It is now even possible to provide guarantees to applications with specific requirements: for instance, \textsc{CloudMirror} \cite{cloudmirror} enables applications to reserve a minimum bandwidth.\par
For the time being, it seems that there is still no valid resource allocation algorithm that takes into account the presence of a network having a data plane that is (in part o completely) capable of basic \gls{inp} operations.
The objective of this thesis is to model and evaluate an \gls{api} through which applications can ask for resources in a data center exploiting \gls{inp} capabilities while providing guarantees (e.g., bandwidth).

\subsection{Modeling \texorpdfstring{\glsentryshort{inp}}{INP} resources}
One of the two goals of this Master's thesis consists in investigating how to model \gls{inp} resources and how to integrate them into RMs.
In order to offer \gls{inp} services to a tenant application, the latter should be able to ask for \gls{inp} resources through an \gls{api}.
To do that, \gls{inp} resources must be modeled not only to support currently existing \gls{inp} solutions such as \cite{daiet} \cite{netchain} \cite{incbricks} \cite{sharp}, but also future ones. 
It may also be convenient to derive a single model to describe both server and \gls{inp} resources.

Classic tenant application requests can often be modeled as a key-value data structure.
CloudMirror \cite{cloudmirror} requires a \gls{tag} as an input, which is a directed graph where each vertex represents an application component and links' weights represent the minimum requested bandwidth.
One possible model could be based on a \gls{tag}, describing network resources and \gls{inp} services as vertexes or links.
Tenant applications could either use the same model used within the data center or a simplified one, adding another level of abstraction.


\subsection{\texorpdfstring{\glsentryshort{inp}}{INP}-aware \texorpdfstring{\glsentrylongpl{rm}}{Resource Managers}} \label{inp_aware_rms}
\glsreset{rm}
In order for everything to work, a network-aware placement algorithm in the \glsentrylong{rm} should be able to consider \gls{inp} and \glslink{resource:logical:server}{server} resources conjunctly: this brings new challenges in the field of resource management as there are currently no \glspl{rm} doing this.
One problem that could arise is due to the fact that \gls{inp} resources are typically very limited in a data center: this thesis argues the importance of an \gls{rm} which is flexible enough to propose alternatives based on the current utilization of \gls{inp} and \glslink{resource:logical:server}{server} resources, since one kind of \glslink{resource:physical}{physical resource} type can become the bottleneck for the other.

\section{Personal contribution and obtained results}

\subsection{System design}

\subsubsection{\texorpdfstring{\Glsentrytext{resource:composite}}{Composite}s translation methods}

\subsection{Resource model}

\subsection{Generic groups}

\subsection{Simulation}

\subsubsection{Results}

\subsection{Conclusions}

% References
\bibliographystyle{abbrv}
\bibliography{../thesis/utility/references}

% Document end
\end{document}